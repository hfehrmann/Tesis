\begin{Lemma}[Substitution]
\label{lemma:wtrans_substitution}
Given a context \Context{} and terms $M$ and $N$, we have:
\begin{center}
    $\WTrans{\subst{M}{x}{N}} \equiv \subst{\WTrans{M}}{x}{\WTrans{N}}$
\end{center}
\end{Lemma}

\begin{Proof}
By induction on $M$ over a generalized context
\begin{ProofCase}{$M \termDef x, M \termDef y$}
These cases are trivial
\end{ProofCase}

\begin{ProofCase}{$M \termDef \Type, M \termDef \Prop$}
These cases are also trivial because $x$ is not involved in the translated term
\end{ProofCase}

\begin{ProofCase}{$M \termDef \app{M_1}{M_2}$}
We have $\WTrans{\subst{(\app{M_1}{M_2})}{x}{N}}$. We get by unfold the definition of substitution and 
the translation
\begin{center}
$\WTrans{\subst{(\app{M_1}{M_2})}{x}{N}}$ 
\\ $\equiv$ \\
$\WTrans{
    \app{(\subst{M_1}{x}{N})}{(\subst{M_2}{x}{N})}
}$
\\ $\equiv$ \\
$\app{
    \WTrans{\subst{M_1}{x}{N}}
}{
    \WTrans{\subst{M_2}{x}{N}}
}
$
\end{center}
It is possible to apply the induction hypothesis over $M_1$ and $M_2$
\begin{center}
$\app{
    (\subst{\WTrans{M_1}}{x}{\WTrans{N}})
}{
    (\subst{\WTrans{M_2}}{x}{\WTrans{N}})
}$
\end{center}
and by folding the definition of substitution and the translation
\begin{center}
$\app{
    (\subst{\WTrans{M_1}}{x}{\WTrans{N}})
}{
    (\subst{\WTrans{M_2}}{x}{\WTrans{N}})
}$
\\ $\equiv$ \\
$\subst{(\app{\WTrans{M_1}}{\WTrans{M_2}})}{x}{\WTrans{N}}$
\\ $\equiv$ \\
$\subst{\WTrans{\app{M_1}{M_2}}}{x}{\WTrans{N}}$
\end{center}
which is the required result.
\end{ProofCase}

\begin{ProofCase}{$M \termDef \abs{x}{A}{M_1}$}
Note that in this case the binder is the same being substituted. By unfolding the definition of
substitution and the translation, we get
\begin{center}
$\WTrans{
    \abs{x}{\subst{A}{x}{N}}{M_1}
}$
\\ $\equiv$ \\
$\abs{x}{\WTypeTrans{\subst{A}{x}{N}}}{\WTrans[\typing{x}{A}]{M_1}}$
\end{center}
It is necessary to do a case analysis over $\ctyping{\subst{A}{x}{N}}{s}$, with $s$ a sort,
to unfold $\WTypeTrans{\subst{A}{x}{N}}$ which could give either:
\begin{itemize}
\item \WTrans{\subst{A}{x}{N}}
\item \app{\El}{\WTrans{\subst{A}{x}{N}}}
\end{itemize}
In both cases, it follows directly that 
$\WTypeTrans{\subst{A}{x}{N}} \equiv \subst{\WTypeTrans{A}}{x}{\WTrans{N}}$
by using the induction hypothesis and unfolding/folding. Then, the conclusion can be established.
\end{ProofCase}

\begin{ProofCase}{$M \termDef \abs{y}{A}{M_1}$}
The binder in this case is distinct from the one being substituted. Nevertheless, it is possible 
to derive, unfolding the substitution just like in the previous case:
\begin{center}
\varabs{y}{\subst{\WTypeTrans{A}}{x}{\WTrans{N}}}{\WTrans[\typing{y}{A}]{\subst{M}{x}{N}}}
\end{center}
Here is possible to apply the induction hypothesis over \WTrans[\typing{y}{A}]{\subst{M}{x}{N}}, 
because the context can be instantiated arbitrarily,
\begin{center}
    \subst{\WTrans[\typing{y}{A}]{M}}{x}{\WTrans[\typing{y}{A}]{N}}
\end{center}
Additionally, we can assume that $y\notin\FreeVar{N}$ due to alpha rename which implies 
\begin{center}
$
\WTrans[\typing{y}{A}]{N}
\equiv
\WTrans{N}
$
\end{center}
because the context is used to type terms on two places and in those places is 
possible to do an inversion over the derivation and \Context{} can be used instead of 
\Context{},\typing{y}{A}. Then, by a folding of the translation we obtain
\begin{center}\subst{\WTrans{\abs{y}{A}{M}}}{x}{\WTrans{N}}\end{center}
\end{ProofCase}

\begin{ProofCase}{$M \termDef \depArrow{y}{A}{B}$}
It is necessary to do a case analysis over \ctyping{\subst{\depArrow{y}{A}{B}}{x}{N}}{s},
for $s$ a sort.
\begin{SubProofCase}{$s \termDef \Prop$}
We have in this case after reduction
\begin{center}
$\depArrow{y}{\WTypeTrans{\subst{A}{x}{N}}}{\WTypeTrans[\typing{y}{A}]{\subst{B}{x}{N}}}$
$\equiv$
$\depArrow{y}{\WTypeTrans{\subst{A}{x}{N}}}{\WTrans[\typing{y}{A}]{\subst{B}{x}{N}}}$
\end{center}
where the last equivalence holds because $\ctyping[\typing{x}{A}]{B}{\Prop}$. In this scenario,
the remaining of the subcase is equal to the lambda case.
\end{SubProofCase}

\begin{SubProofCase}{$s \termDef \Type$}
We have in this case after reduction
\begin{center}
$\app{\TypeVal}{M_1}{M_2}$
\end{center}
with
\begin{itemize}
\item $M_1 \termDef \depArrow{y}{\WTypeTrans{\subst{A}{x}{N}}}{
                            \WTypeTrans[\typing{y}{A}]{\subst{B}{x}{N}}}$
\item $M_2 \termDef \abs{e}{\ExceptionType}{y}{\WTypeTrans{\subst{A}{x}{N}}}{
                            \app{\WTransErr[\typing{y}{A}]{\subst{B}{x}{N}}}{e}}$
\end{itemize}
Both cases were solved previously by an unfolding, induction 
hypothesis application and folding again.
\end{SubProofCase}
Note that here no distinction were made between the binder being substituted and the one binding.
That is because the treatment is similar to what we did in the lambda case
\end{ProofCase}
\end{Proof}