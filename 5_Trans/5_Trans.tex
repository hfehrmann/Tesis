\chapter{New Parametric Exceptional Translation}
\label{ch:translation}

%\setcounter{page}{1}
%\setcounter{figure}{0}
%\setcounter{globDefs}{0}

\newcommand{\FreeVar}[1]{\mathtexttt{FreeVar}(#1)}

\newcommand{\ExceptionType}{\ensuremath{\mathbb{E}}}
\newcommand{\BExceptionType}{\text{\rmfamily \textbf{E}}}

\newcommand{\raiseFun}{\mathtexttt{raise}}
\newcommand{\unit}{\mathtexttt{unit}}
\newcommand{\ttUnit}{\mathtexttt{tt}}
\newcommand{\eqRefl}{\mathtexttt{refl}}


\renewcommand{\CicType}{\textsc{Type}}
\renewcommand{\CicProp}{\textsc{Prop}}
\newcommand{\CicWeak}{\textsc{Weakening}}
\newcommand{\CicTypeProd}{\textsc{Type-Prod}}
\newcommand{\CicPropProd}{\textsc{Prop-Prod}}
\newcommand{\CicImpred}{\textsc{Impred}}
\renewcommand{\CicConv}{\textsc{Conv}}
\newcommand{\CicAbs}{\textsc{Abs}}
\renewcommand{\CicApp}{\textsc{App}}
\newcommand{\CicIdem}{\textsc{Idem}}
\newcommand{\CicWfnil}{\textsc{Wf-Empty}}
\newcommand{\CicWfcons}{\textsc{Wf-Cons}}

In \todo{put ref} is presented a syntactic translation for \CIC{} that adds exception to the 
language. This translation breaks the logic consistency of the calculus. In \todo{put ref},
the logic consistency is recovered with parametricity techniques if the exceptions do not
flow at the top level.

The calculus used for the translation is not the same presented here but a variation. The 
variations only take part on the functional fragment of the calculus (calculus without the inductives).
We present it in \figRef{fig:CCw*}

\begin{figure}
    \begin{mathpar}
    A,B,N,M,R,S\ \termDef\ \Type\ |\ \Prop\ |\ x\ |\ \app{M}{N}\ |\ \abs{x}{A}{M}\ |\ \vardepArrow{A}{B} \\
    \Context,\varContext\ \termDef\ \EContext\ |\ \Context,\typing{x}{A} \\
    
    \inferrule*[lab=\CicType]{\wfContext{\Context} \and i<j}{\ctyping{\Type}{\Type[j]}}
    \and
    \inferrule*[lab=\CicProp]{\wfContext{\Context}}{\ctyping{\Prop}{\Type}}
    \and  
    \inferrule*[lab=\CicWeak]{\ctyping{M}{B} \and \and \ctyping{A}{s}}{\ctyping[\typing{a}{A}]{M}{B}}
    
    \\ 
    
    \inferrule*[lab=\CicTypeProd]{\ctyping{A}{\Type} \and \ctyping[\typing{x}{A}]{B}{\Type[j]}}{%
               \ctyping{\vardepArrow{A}{B}}{\Type[\text{max(i,j)}]}}
    \and
    \inferrule*[lab=\CicPropProd]{\ctyping{A}{\Prop} \and \ctyping[\typing{x}{A}]{B}{\Type}}{%
               \ctyping{\vardepArrow{A}{B}}{\Type}}
    
    \\
    
    \inferrule*[lab=\CicImpred]{\ctyping{A}{s} \and \ctyping[\typing{x}{A}]{B}{\Prop}}{%
               \ctyping{\vardepArrow{A}{B}}{\Prop}}
               
    \inferrule*[lab=\CicConv]{\ctyping{M}{B} \and \ctyping{A}{s} \and \convertible{A}{B}}{\ctyping{M}{A}}
    
    \\
    
    \inferrule*[lab=\CicAbs]{\ctyping[\typing{x}{A}]{M}{B} \and \ctyping{\vardepArrow{A}{B}}{s}}{%
               \ctyping{\abs{x}{A}{M}}{\vardepArrow{A}{B}}}
    \and
    \inferrule*[lab=\CicApp]{\ctyping{M}{\vardepArrow{A}{B} \and \ctyping{N}{A}}}{\ctyping{\app{M}{N}}{\subst{B}{x}{N}}}
    
    \\
    
    \inferrule*[lab=\CicWfnil]{ }{\wfContext{\EContext}} 
    \and 
    \inferrule*[lab=\CicWfcons]{\ctyping{A}{s}}{\wfContext{\Context,\typing{x}{A}}}
    \and
    \inferrule*[lab=\CicIdem]{\ctyping{A}{s}}{\ctyping[\typing{x}{A}]{x}{A}}
    
    \\
    
    \inferrule{ }{\convertible{\app{(\abs{x}{A}{M})}{N}}{\subst{M}{x}{N}}}
    \and
    \inferrule{ }{\convertible{M}{\abs{x}{A}{(\app{M}{x})}}}
    \\
    \inferrule{\convertible{M_1}{M_2}}{\convertible{\app{M_1}{N}}{\app{M_2}{N}}}
    \and
    \inferrule{\convertible{N_1}{N_2}}{\convertible{\app{M}{N_1}}{\app{M}{N_2}}}
    
    \end{mathpar}
    \caption{\CCw + \Prop}
    \label{fig:CCw*}
\end{figure}

% The exceptional translation is a syntactical translation of \CIC{} that adds exceptions to the language. 
% The formalization is done in two stages. First, over a \CCw{}, which is similar to \CC{} with a cumulative 
% universe instead of the impredicative, and 
% then over the same language adding inductives. The underlying calculus \CCw{} is presented in \figRef{fig:CCw*}, 
% with the impredicative sort added. Here we have a set $\Sort\termEq \{\Prop\}\cup\{\Type{} |\ i\in\Nat\}$ and 
% $s\in\Sort$ for every $s$ in the definition.

From now on we define \Theory{} a theory that interprets \CIC{}, \ETheory{} the exceptional translation of 
\Theory{} and \PTheory{} the parametric exceptional translation of \Theory{}.

In \ETheory{} it is possible to define a new type \BExceptionType{} which exposes the underlying 
exceptional type \ExceptionType{} of the translation.
\begin{center} 
$\Trans{\BExceptionType}\termDef \app{\TypeVal}{\ExceptionType}{(\abs{e}{\ExceptionType}{e})}$ 
\end{center}
With this type, it is possible to define in \ETheory{} a function that inhabits any type providing a term
belonging to the 
exception $\typing{\raiseFun{}}{\depArrow{A}{\Type[]{}}{\arrow{\BExceptionType}{A}}}$ by:
\begin{center}
$\Trans{\raiseFun}\termDef \abs{A}{\TypeTrans{\Type[]{}}}{e}{\ExceptionType}{\app{\Err}{A}{e}}$ 
\end{center}
However, these previously defined terms do not belong to \PTheory, because both terms do not admit a 
parametric translation over \ParamTrans{\cdot}. Intuitively, \ParamTrans{\cdot} recursively verifies that 
a term is composed only by pure terms. Thus, it is not possible to establish parametric proofs that use 
impure programs, ie, these terms belong only to \ETheory{} but not to \PTheory. In the following 
example $\ExceptionType\termDef\unit$:
\begin{center}
$\exists (n:\mathtexttt{nat}),\ (\app{\raiseFun}{\ttUnit}) = n $
\end{center}
Given the definition of equality, existencial and natural in \figRef{fig:EqAndExDefs}, the proposition admits two
proofs in \ETheory{}:
\begin{center}
$(\app{\mathtexttt{intro}}{\mathtexttt{nat}}{(\abs{n}{\mathtexttt{nat}}{(\app{\raiseFun}{\ttUnit}) = n})}{
      (\app{\raiseFun}{\ttUnit})}{\eqRefl})$
\hspace{2em} 
$(\app{\raiseFun}{\ttUnit})$
\end{center}
It is desirable that the parametric translation allows the former, but prohibits the latter because 
the first establishes a property for the exceptional term while the second is an impure proof. Currently,
both terms are rejected because $(\app{\raiseFun}{\ttUnit})$ is used. 

\begin{figure}
\begin{CoqDef}%{\par\bfseries\ttfamily\noindent%
\IndIntro[2][\typing{\Indeq}{\depArrow{A}{\Type}{x}{A}{\arrow{A}{\Prop}}}][
             \typing{\Consrefl}{\depArrow{A}{\Type}{x}{A}{\app{\Indeq}{A}{x}{x}}}]

\noindent
\IndIntro[2][\typing{\Index}{\depArrow{A}{\Type}{P}{\arrow{A}{\Prop}}{\Prop}}][
             \typing{\Consintro}{\depArrow{A}{\Type}{P}{\arrow{A}{\Prop}}{
                                           \depArrow{x}{A}{p}{\app{P}{x}}{\app{\Index}{A}{P}}}}]
\end{CoqDef}%}%
    \caption{Inductive definitions}
    \label{fig:EqAndExDefs}
\end{figure}

Here we present a new syntactical 
translation over \CIC{} which is a variant of the exceptional one,
the \emph{weakly parametric exceptional translation} \WParamTrans{\cdot},
that relaxes the translation
over propositions and allows reasoning over impure programs, but prohibits impure proofs, \ie{}, 
accept the first proof and reject the second. 

\section{Translation for the Functional Fragment}

The exceptional translation transforms the sort hierarchy to a type hierarchy with a default exceptional
constructor:
\begin{itemize}
    \item $\Trans{\Type} \equiv \app{\TypeVal}{ \type}{\TypeErr} $
    \item $\Trans{\Prop} \equiv \app{\TypeVal}{ \prop}{\PropErr} $
\end{itemize}
\newcommand{\wSymbol}{\ensuremath{\omega}}
\NewDocumentCommand{\WTrans}{D<>{\Context}om}{%
    \ensuremath{[#3]^{#1\IfValueT{#2}{,#2}}_{\wSymbol}}%
}
\NewDocumentCommand{\WTypeTrans}{D<>{\Context}om}{%
    \ensuremath{[\![#3]\!]^{#1\IfValueT{#2}{,#2}}_{\wSymbol}}%
}
\NewDocumentCommand{\WTransErr}{D<>{\Context}om}{%
    \ensuremath{[#3]^{#1\IfValueT{#2}{,#2}}_{\wSymbol\,\phi}}%
}
\NewDocumentCommand{\WContextTrans}{m}{\WTypeTrans<>{#1}}
\renewcommand{\WTermTrans}[1]{\ensuremath{#1_{\wSymbol}}}

Both sorts can be inhabited by an exception in \ETheory{} providing an inhabitant
of \ExceptionType{}. We want to enforce the purity of the propositions, so the new translation
\WTrans{\cdot}
treats the propositional sort a little different 
\begin{center}
$\WTrans<>{\Prop} \equiv \app{\TypeVal}{\Prop}{(\varabs{_}{\ExceptionType}{\depArrow{A}{\Prop}{A}})} $
\end{center}
The default exceptional constructor for this case is chosen degenerated because in essence is never used in a 
meaningful way. Note that this new translation is not homogeneous, in the sense that 
the translation for the sorts differ in a fundamental way. As a consequence, it is necessary to tag
every term with the sort they belong to. This can be achieved by keeping an environment along the
translation. With this change, the new translation is defined in 
\figRef{fig:weaklyParExcTransDef}.

\begin{figure}
    \centering
    \begin{minipage}{\textwidth}
    \renewcommand{\arraystretch}{1.15}
    \begin{tabular}{l@{$\hspace{0.1em}\termDef$\ \ }l}
    \WTrans{\Type} & $\app{\TypeVal}{\type}{\TypeErr}$ \\
    \WTrans{\Prop} & $\app{\TypeVal}{
                           \Prop}{
                           (\abs{\noBinder}{\ExceptionType}{ (\depArrow{A}{\Prop}{A}) })}$ \\
    \WTrans{x} & $x$ \\
     
    \WTrans{\abs{x}{A}{M}} & $\abs{x}{\WTypeTrans{A}}{\WTrans[\typing{x}{A}]{M}}$ \\
    \WTrans{\app{\M{}}{\N{}}} & $\app{\WTrans{M}}{\WTrans{N}}$ \\
    \WTrans{\depArrow{x}{A}{B}} & $\! \! \left \{ \begin{matrix*}[l]
                                      \depArrow{x}{\WTypeTrans{A}}{\WTypeTrans[\typing{x}{A}]{B}}
                                       &
                                            \ctyping{\depArrow{x}{A}{B}}{\Prop} \\
                                      \app{\TypeVal}{
                                        (\depArrow{x}{\WTypeTrans{A}}{\WTypeTrans[\typing{x}{A}]{B}})
                                      }{
                                        (\abs{e}{\ExceptionType}{x}{\WTypeTrans{A}}{
                                              \app{\WTransErr[\typing{x}{A}]{B}}{e}})
                                      }
                                        & \otherwise
                                      \end{matrix*} \right.$ \\
    \WTransErr{A} & \app{\Err}{\WTrans{A}} \\
    \WTypeTrans{A} & $\! \! \left \{ \begin{matrix*}[l]
                                     \WTrans{A} & \ctyping{A}{\Prop} \\
                                      \app{\El}{\WTrans{A}} & \otherwise
                                      \end{matrix*} \right.$ \\
    \hline
    \rule{0pt}{15pt} \WContextTrans{\EContext} & \EContext \\
    \WContextTrans{\Context, \typing{x}{A}} &     
                $\WContextTrans{\Context},\typing{x}{\WTypeTrans<\Context>{A}}$
    \end{tabular}
    \end{minipage}
    \caption{Weakly Parametric Exceptional Translation}
    \label{fig:weaklyParExcTransDef}
\end{figure}

The translation is composed by four parts: \WTrans{\cdot} is used at the \emph{term} level, 
\WTypeTrans{\cdot} at the type level, \WContextTrans{\cdot} for contexts and \WTransErr{\cdot} for 
generating the exceptional default constructor. In figure \figRef{fig:translation_difference}
we can see the difference between both translations.
Note that reductions are applied in figure in order to be more concise.
Here we can see that if we restrict the calculus with only the \Type[]{} sort, then both translations
are equivalent. The translation remains almost unchanged with the new translation of \Prop{}. The interesting 
case is when the term being translated types to \Prop{}, which correspond to the last example in both examples.
The term and type translation are equal is this case (this are the branches in 
\figRef{fig:weaklyParExcTransDef}). Essentially, here is were the translation enforces the fact that 
propositions must remains pure in the exceptional layer. If a term types to \Prop{}, 
then in the translation 
it will also translate to \Prop{}, where no special machinery is defined to be able to raise an exception. 
Note that the \raiseFun{} term uses the second argument of \TypeVal{} to inhabit any type but for \Prop{}
such function does not exist.

\begin{figure}
\def\arraystretch{1.5}
\centering
\subfloat[][Translation for \Trans{\cdot}/\TypeTrans{\cdot}]{
\begin{tabular}{lll}
    \cline{2-3}
    & \Trans{\cdot} & \TypeTrans{\cdot} \\
    \hline
    % --------------------------
    \Type
    & 
    \app{\TypeVal}{\type}{\TypeErr}
    &
    \type \\
    % --------------------------
    \Prop
    & 
    \app{\TypeVal}{\prop}{\PropErr}
    &
    \prop \\
    % --------------------------
    \arrow{\Type}{\Type[j]}
    & 
    \app{\TypeVal}{
        (\arrow{\type}{\type[j]})}{
        (\abs{e}{\ExceptionType}{t}{\type}{\TypeErr})} 
    &
    \arrow{\type}{\type[j]} \\
    % --------------------------
    \arrow{\Type}{\Prop}
    & 
    \app{\TypeVal}{
        (\arrow{\type}{\prop})}{
        (\abs{e}{\ExceptionType}{t}{\type}{\PropErr})}
    &
    \arrow{\type}{\prop} \\
    % --------------------------
    \depArrow{A}{\Prop}{A}
    & 
    \app{\TypeVal}{
        (\depArrow{A}{\prop}{A})}{
        (\abs{e}{\ExceptionType}{A}{\prop}{\PropErr})} 
    &
    \depArrow{A}{\prop}{A} \\
    \hline
\end{tabular}
}

\subfloat[][Translation for \WTrans{\cdot}/\WTypeTrans{\cdot}]{
\begin{tabular}{lll}
    \cline{2-3}
    & \WTrans{\cdot} & \WTypeTrans{\cdot} \\
    \hline
    % --------------------------
    \Type
    & 
    \app{\TypeVal}{\type}{\TypeErr}
    &
    \type \\
    % --------------------------
    \Prop
    & 
    \app{\TypeVal}{\Prop}{(\depArrow{A}{\Prop}{A})}
    &
    \Prop \\
    % --------------------------
    \arrow{\Type}{\Type[j]}
    & 
    \app{\TypeVal}{
        (\arrow{\type}{\type[j]})}{
        (\abs{e}{\ExceptionType}{t}{\type}{\TypeErr})} 
    &
    \arrow{\type}{\type[j]} \\
    % --------------------------
    \arrow{\Type}{\Prop}
    &
    \app{\TypeVal}{
        (\arrow{\type}{\Prop})}{
        (\abs{e}{\ExceptionType}{t}{\type}{\depArrow{A}{\Prop}{A}})} 
    & 
    \arrow{\type}{\Prop} \\
    % --------------------------
    \depArrow{A}{\Prop}{A} 
    & 
    \depArrow{A}{\Prop}{A}
    &
    \depArrow{A}{\Prop}{A} \\
    \hline
\end{tabular}
}
    \caption{Difference between the translations}
    \label{fig:translation_difference}
\end{figure}

\begin{Lemma}
\label{lemma:wtrans_sort_involutive}
Given a context \Context{}, a term $M$ and a sort $s$ such that 
\ctyping[][\WContextTrans{\Context}]{\WTrans{M}}{\WTypeTrans{s}}, then 
\ctyping[][\WContextTrans{\Context}]{\WTypeTrans{M}}{s}.
\end{Lemma}

\begin{Proof}
By a case analysis over $s$.
\begin{ProofCase}{$s \termDef \Type$}
After reduction, we have to prove $\ctyping[][\WContextTrans{\Context}]{\app{\El}{\WTrans{M}}}{s}$
given $\ctyping[][\WContextTrans{\Context}]{\WTrans{M}}{\type}$. Because \El{} has type
$\arrow{\type}{\Type}$, the result is obtained by an application of the \CicApp{} rule.
\end{ProofCase}

\begin{ProofCase}{$s \termDef Prop$}
After reduction, we have to prove $\ctyping[][\WContextTrans{\Context}]{\WTrans{M}}{\Prop}$
given $\ctyping[][\WContextTrans{\Context}]{\WTrans{M}}{\Prop}$, which is trivial.
\end{ProofCase}
\end{Proof}

\begin{Lemma}[Substitution]
\label{lemma:wtrans_substitution}
Given a context \Context{} and terms $M$ and $N$, we have:
\begin{center}
    $\WTrans{\subst{M}{x}{N}} \equiv \subst{\WTrans{M}}{x}{\WTrans{N}}$
\end{center}
\end{Lemma}

\begin{Proof}
By induction on $M$ over a generalized context
\begin{ProofCase}{$M \termDef x, M \termDef y$}
These cases are trivial
\end{ProofCase}

\begin{ProofCase}{$M \termDef \Type, M \termDef \Prop$}
These cases are also trivial because $x$ is not involved in the translated term
\end{ProofCase}

\begin{ProofCase}{$M \termDef \app{M_1}{M_2}$}
We have $\WTrans{\subst{(\app{M_1}{M_2})}{x}{N}}$. We get by unfold the definition of substitution and 
the translation
\begin{center}
$\WTrans{\subst{(\app{M_1}{M_2})}{x}{N}}$ 
\\ $\equiv$ \\
$\WTrans{
    \app{(\subst{M_1}{x}{N})}{(\subst{M_2}{x}{N})}
}$
\\ $\equiv$ \\
$\app{
    \WTrans{\subst{M_1}{x}{N}}
}{
    \WTrans{\subst{M_2}{x}{N}}
}
$
\end{center}
It is possible to apply the induction hypothesis over $M_1$ and $M_2$
\begin{center}
$\app{
    (\subst{\WTrans{M_1}}{x}{\WTrans{N}})
}{
    (\subst{\WTrans{M_2}}{x}{\WTrans{N}})
}$
\end{center}
and by folding the definition of substitution and the translation
\begin{center}
$\app{
    (\subst{\WTrans{M_1}}{x}{\WTrans{N}})
}{
    (\subst{\WTrans{M_2}}{x}{\WTrans{N}})
}$
\\ $\equiv$ \\
$\subst{(\app{\WTrans{M_1}}{\WTrans{M_2}})}{x}{\WTrans{N}}$
\\ $\equiv$ \\
$\subst{\WTrans{\app{M_1}{M_2}}}{x}{\WTrans{N}}$
\end{center}
which is the required result.
\end{ProofCase}

\begin{ProofCase}{$M \termDef \abs{x}{A}{M_1}$}
Note that in this case the binder is the same being substituted. By unfolding the definition of
substitution and the translation, we get
\begin{center}
$\WTrans{
    \abs{x}{\subst{A}{x}{N}}{M_1}
}$
\\ $\equiv$ \\
$\abs{x}{\WTypeTrans{\subst{A}{x}{N}}}{\WTrans[\typing{x}{A}]{M_1}}$
\end{center}
It is necessary to do a case analysis over $\ctyping{\subst{A}{x}{N}}{s}$, with $s$ a sort,
to unfold $\WTypeTrans{\subst{A}{x}{N}}$ which could give either:
\begin{itemize}
\item \WTrans{\subst{A}{x}{N}}
\item \app{\El}{\WTrans{\subst{A}{x}{N}}}
\end{itemize}
In both cases, it follows directly that 
$\WTypeTrans{\subst{A}{x}{N}} \equiv \subst{\WTypeTrans{A}}{x}{\WTrans{N}}$
by using the induction hypothesis and unfolding/folding. Then, the conclusion can be established.
\end{ProofCase}

\begin{ProofCase}{$M \termDef \abs{y}{A}{M_1}$}
The binder in this case is distinct from the one being substituted. Nevertheless, it is possible 
to derive, unfolding the substitution just like in the previous case:
\begin{center}
\varabs{y}{\subst{\WTypeTrans{A}}{x}{\WTrans{N}}}{\WTrans[\typing{y}{A}]{\subst{M}{x}{N}}}
\end{center}
Here is possible to apply the induction hypothesis over \WTrans[\typing{y}{A}]{\subst{M}{x}{N}}, 
because the context can be instantiated arbitrarily,
\begin{center}
    \subst{\WTrans[\typing{y}{A}]{M}}{x}{\WTrans[\typing{y}{A}]{N}}
\end{center}
Additionally, we can assume that $y\notin\FreeVar{N}$ due to alpha rename which implies 
\begin{center}
$
\WTrans[\typing{y}{A}]{N}
\equiv
\WTrans{N}
$
\end{center}
because the context is used to type terms on two places and in those places is 
possible to do an inversion over the derivation and \Context{} can be used instead of 
\Context{},\typing{y}{A}. Then, by a folding of the translation we obtain
\begin{center}\subst{\WTrans{\abs{y}{A}{M}}}{x}{\WTrans{N}}\end{center}
\end{ProofCase}

\begin{ProofCase}{$M \termDef \depArrow{y}{A}{B}$}
It is necessary to do a case analysis over \ctyping{\subst{\depArrow{y}{A}{B}}{x}{N}}{s},
for $s$ a sort.
\begin{SubProofCase}{$s \termDef \Prop$}
We have in this case after reduction
\begin{center}
$\depArrow{y}{\WTypeTrans{\subst{A}{x}{N}}}{\WTypeTrans[\typing{y}{A}]{\subst{B}{x}{N}}}$
$\equiv$
$\depArrow{y}{\WTypeTrans{\subst{A}{x}{N}}}{\WTrans[\typing{y}{A}]{\subst{B}{x}{N}}}$
\end{center}
where the last equivalence holds because $\ctyping[\typing{x}{A}]{B}{\Prop}$. In this scenario,
the remaining of the subcase is equal to the lambda case.
\end{SubProofCase}

\begin{SubProofCase}{$s \termDef \Type$}
We have in this case after reduction
\begin{center}
$\app{\TypeVal}{M_1}{M_2}$
\end{center}
with
\begin{itemize}
\item $M_1 \termDef \depArrow{y}{\WTypeTrans{\subst{A}{x}{N}}}{
                            \WTypeTrans[\typing{y}{A}]{\subst{B}{x}{N}}}$
\item $M_2 \termDef \abs{e}{\ExceptionType}{y}{\WTypeTrans{\subst{A}{x}{N}}}{
                            \app{\WTransErr[\typing{y}{A}]{\subst{B}{x}{N}}}{e}}$
\end{itemize}
Both cases were solved previously by an unfolding, induction 
hypothesis application and folding again.
\end{SubProofCase}
Note that here no distinction were made between the binder being substituted and the one binding.
That is because the treatment is similar to what we did in the lambda case
\end{ProofCase}
\end{Proof}

\begin{Lemma}[Conversion]
\label{lemma:wtrans_conversion}
Given a context \Context{} and terms $M$ and $N$, if \convertible{M}{N} then 
\convertible{\WTrans{M}}{\WTrans{N}}
\end{Lemma}

\begin{Proof}
By induction over the conversion.
\begin{ProofCase}{\betaRed{}}
Here we have
\begin{itemize}
\item $M \termDef \app{(\abs{x}{A}{M_1})}{N_1}$
\item $N \termDef \subst{M_1}{x}{N_1}$
\end{itemize}
Unfolding the translation and applying \lemmaRef{lemma:wtrans_substitution}:
\begin{itemize}
\item $\app{(\abs{x}{\WTypeTrans{A}}{
                    \WTrans[\typing{x}{A}]{M_1}})}{\WTrans{N_1}}$
\item $\subst{\WTrans[\typing{x}{A}]{M_1}}{x}{\WTrans{N}}$
\end{itemize}
With these terms is possible to establish conversion between the two.
\end{ProofCase}

\begin{ProofCase}{\etaExp{}}
Here we have
\begin{itemize}
\item $M \termDef M_1$
\item $N \termDef \abs{x}{A}{(\app{M_1}{x})}$
\end{itemize}
Unfolding the translation:
\begin{itemize}
\item $\WTrans{M_1}$
\item $\abs{x}{\WTypeTrans{A}}{(\app{\WTrans{M_1}}{x})}$
\end{itemize}
Same as before, it is possible to establish the conversion between the two terms.
\end{ProofCase}

\begin{ProofCase}{App}
Without loss of generality, we can set that the conversion is established for 
left term of the application. The other case is similar. We have
\begin{itemize}
\item $M \termDef \app{M_1}{N'}$
\item $N \termDef \app{M_2}{N'}$
\end{itemize}
with \convertible{M_1}{M_2}. Unfolding the translation we get:
\begin{itemize}
\item $\app{\WTrans{M_1}}{\WTrans{N'}}$
\item $\app{\WTrans{M_2}}{\WTrans{N'}}$
\end{itemize}
Applying the induction hypothesis allows the conclusion.
\end{ProofCase}

\end{Proof}

The typing soundness theorem is, taking in consideration the previous discussion:
\begin{Theorem}
Given \Context, $M$, and $A$ with \wfContext{\Context} and \ctyping{M}{A}, then we have
\ctyping[][\WContextTrans{\Context}]{\WTrans{M}}{\WTypeTrans{A}}
\end{Theorem}

\begin{Proof}
By a mutual induction over the typing derivation.
\begin{ProofCase}{\CicType}
We need to prove 
$\ctyping[][\WContextTrans{\Context}]{
          \app{\TypeVal}{\type}{\TypeErr}}{
          \type[j]}$. 
We now that \TypeVal{} has type 
$\depArrow{A}{\type[i']}{\arrow{\arrow{\ExceptionType}{A}}{\type[i']}}$ and \type{} has type 
$\Type[j]$ with $i<j$. By a double \CicApp{}, we can conclude the result unifying $j$ with $i'$.
\end{ProofCase}

\begin{ProofCase}{\CicProp}
We need to prove 
$\ctyping[][\WContextTrans{\Context}]{
          \app{\TypeVal}{\Prop}{(\depArrow{A}{\Prop}{A})}}{
          \type}$. 
This is similar to the previous case.
\end{ProofCase}

\begin{ProofCase}{\CicWeak}
We need to prove 
$\ctyping[\typing{a}{\WTypeTrans{A}}][\WContextTrans{\Context}]{
          \WTrans{M}}{
          \WTypeTrans{B}}$. 
By the induction hypothesis we have $\ctyping[][\WContextTrans{\Context}]{
          \WTrans{M}}{
          \WTypeTrans{B}}$. It only remains to show
$\ctyping[][\WContextTrans{\Context}]{\WTypeTrans{A}}{s'}$ for some $s'$. We proceed by 
a case analysis over $s$
\begin{SubProofCase}{$s \equiv \Prop{}$}
We have $\ctyping[][\WContextTrans{\Context}]{\WTrans{A}}{\Prop}$ by the induction hypothesis
and also $\WTypeTrans{A}\equiv\WTrans{A}$. We get the result by applying \CicWeak{} with 
$s'\equiv\Prop$.
\end{SubProofCase}

\begin{SubProofCase}{$s \equiv \Type$}
We have $\ctyping[][\WContextTrans{\Context}]{\WTrans{A}}{\type[s']}$ by the induction hypothesis
and also $\WTypeTrans{A}\equiv\app{\El}{\WTrans{A}}$. In order to conclude, we need that 
$\ctyping[][\WContextTrans{\Context}]{\app{\El}{\WTrans{A}}}{s'}$. Recall that
\El{} has type \arrow{\type}{\Type}. By \CicApp{} on \app{\El}{\WTrans{A}}, we derive it 
types to \Type[s']. It is possible to conclude by \CicWeak{} on the previous results unifying
the universe levels.
\end{SubProofCase}
\end{ProofCase}

\begin{ProofCase}{\CicTypeProd}
We need to prove 
\begin{center}
$\ctyping[][\WContextTrans{\Context}]{
    \app{\TypeVal}{
        (\depArrow{x}{\WTypeTrans{A}}{\WTypeTrans[\typing{x}{A}]{B}}) 
        }{
        (\abs{e}{\ExceptionType}{x}{\WTypeTrans{A}}{\app{\WTransErr[\typing{x}{A}]{B}}{e}})
        }
    }{
    \type[max(i,j)]
    }$. 
\end{center}
If the first argument of \TypeVal{} types to $\type[max(i,j)]$ and the second to 
$(\depArrow{x}{\WTypeTrans{A}}{\WTypeTrans[\typing{x}{A}]{B}})$, we can conclude the proof
with a double application of \CicApp{} rule which gives the desired type. We proceed to 
check the first argument.

\noindent By the induction hypothesis we know that:
\begin{itemize}
\item $\ctyping[][\WContextTrans{\Context}]{\WTrans{A}}{\type}$
\item $\ctyping[\typing{x}{\WTypeTrans{A}}][\WContextTrans{\Context}]{
                \WTrans[\typing{x}{A}]{B}}{\type[j]}$
\end{itemize}
We can derive from these hypothesis the followings:
\begin{itemize}
\item $\ctyping[][\WContextTrans{\Context}]{\WTypeTrans{A}}{\Type}$
\item $\ctyping[\typing{x}{\WTypeTrans{A}}][\WContextTrans{\Context}]{
                \WTypeTrans[\typing{x}{A}]{B}}{\Type[j]}$
\end{itemize}
by expanding the definition of \WTypeTrans{\cdot} and applying \CicApp{} rule
over \El{} and \WTrans{\cdot} in both cases. By \CicTypeProd{} rule we get
\begin{center}
$\ctyping[][\WContextTrans{\Context}]{
    \depArrow{x}{\WTypeTrans{A}}{\WTypeTrans[\typing{x}{A}]{B}}}{
    \Type[max(i,j)]}$
\end{center}
It remains to show that 
\begin{center}
$\ctyping[][\WContextTrans{\Context}]{
    (\abs{e}{\ExceptionType}{x}{\WTypeTrans{A}}{\app{\WTransErr[\typing{x}{A}]{B}}{e}})
    }{
    (\arrow{\ExceptionType}{\depArrow{x}{\WTypeTrans{A}}{\WTypeTrans[\typing{x}{A}]{B}}})
}$
\end{center}
which can be reduced to prove the body of the lambda
\begin{center}
$\ctyping[\typing{e}{\ExceptionType},\typing{x}{\WTypeTrans{A}}][\WContextTrans{\Context}]{
    (\app{\WTransErr[\typing{x}{A}]{B}}{e})
    }{
    \WTypeTrans[\typing{x}{A}]{B}
}$
\end{center}
We have $\app{\WTransErr[\typing{x}{A}]{B}}{e} \equiv \app{\Err}{\WTrans[\typing{x}{A}]{B}}{e}$ 
by definition, and $\typing{\Err}{\depArrow{A}{\type}{\arrow{\ExceptionType}{\app{\El}{A}}}}$.
Applying \CicApp{} two times we get 
\begin{center}
$\ctyping[\typing{e}{\ExceptionType},\typing{x}{\WTypeTrans{A}}][\WContextTrans{\Context}]{
    (\app{\Err}{\WTrans[\typing{x}{A}]{B}}{e})
    }{
    (\app{\El}{\WTrans[\typing{x}{A}]{B}})
}$
\end{center}
and noticing that $\app{\El}{\WTrans[\typing{x}{A}]{B}} \equiv \WTypeTrans[\typing{x}{A}]{B}$
we are able to conclude.

The first and second argument of \TypeVal{} type to expected type which allow us to conclude.
\end{ProofCase}

\begin{ProofCase}{\CicPropProd}
This case is similar to the previous one with the difference that now we have
$\ctyping[][\WContextTrans{\Context}]{\WTypeTrans{A}}{\Prop}$ instead of
$\ctyping[][\WContextTrans{\Context}]{\WTypeTrans{A}}{\Type}$. This holds
because $\WTypeTrans{A} \equiv \WTrans{A}$ 
and $\ctyping[][\WContextTrans{\Context}]{\WTrans{A}}{\Prop}$ holds by the induction
hypothesis. \\
The rest of the proof remains almost unchanged.
\end{ProofCase}

\begin{ProofCase}{\CicImpred}
In this case we need to prove
\begin{center}
$\ctyping[][\WContextTrans{\Context}]{
    (\depArrow{x}{\WTypeTrans{A}}{\WTypeTrans[\typing{x}{A}]{B}})
    }{
    \Prop
    }$. 
\end{center}
because $\ctyping{(\depArrow{x}{A}{B})}{\Prop}$. By the induction hypothesis we have
\begin{itemize}
\item $\ctyping[][\WContextTrans{\Context}]{\WTrans{A}}{\WTypeTrans{s}}$
\item $\ctyping[\typing{x}{\WTypeTrans{A}}][\WContextTrans{\Context}]{
                \WTrans[\typing{x}{A}]{B}}{\Prop}$
\end{itemize}
It is possible to derive $\ctyping[][\WContextTrans{\Context}]{\WTypeTrans{A}}{s}$ by 
\lemmaRef{lemma:wtrans_sort_involutive}.
With the previous hypothesis, the desired result is obtained by a direct application of 
\CicImpred{} rule.
\end{ProofCase}

\begin{ProofCase}{\CicAbs}
We need to prove
\begin{center}
$\ctyping[][\WContextTrans{\Context}]{
    (\abs{x}{\WTypeTrans{A}}{\WTrans[\typing{x}{A}]{M}})
    }{
    (\depArrow{x}{\WTypeTrans{A}}{\WTypeTrans[\typing{x}{A}]{B}})
    }$. 
\end{center}
The induction hypothesis gives us
\begin{itemize}
\item $\ctyping[\typing{x}{\WTypeTrans{A}}][\WContextTrans{\Context}]{
        \WTrans{M}}{\WTypeTrans{B}}$
\item $\ctyping[][\WContextTrans{\Context}]{
        \WTrans{\depArrow{x}{A}{B}}}{\WTypeTrans{s}}$
\end{itemize}
We can derive 
$\ctyping[][\WContextTrans{\Context}]{\WTypeTrans{\depArrow{x}{A}{B}}}{s}$ by  
\lemmaRef{lemma:wtrans_sort_involutive}. Noticing that
\begin{center}
$
\ctyping[][\WContextTrans{\Context}]{\WTypeTrans{\depArrow{x}{A}{B}}}{s}
\equiv
\ctyping[][\WContextTrans{\Context}]{\depArrow{x}{\WTypeTrans{A}}{\WTypeTrans{B}}}{s}
$
\end{center}
the previous hypotheses allow us to obtain the result by an application of \CicAbs{} rule.
\end{ProofCase}

\begin{ProofCase}{\CicApp}
Here we need to prove
\begin{center}
$\ctyping[][\WContextTrans{\Context}]{\WTrans{\app{M}{N}}}{\WTypeTrans{\subst{M}{x}{N}}}$
\end{center}
By an unfolding the translation and \lemmaRef{lemma:wtrans_substitution}
\begin{center}
$\ctyping[][\WContextTrans{\Context}]{
    \app{\WTrans{M}}{\WTrans{N}}}{\subst{\WTypeTrans{B}}{x}{\WTrans{N}}}$
\end{center}
Also, we have as induction hypothesis
\begin{itemize}
\item \ctyping[][\WContextTrans{\Context}]{\WTrans{M}}{\depArrow{x}{\WTypeTrans{A}}{\WTypeTrans{B}}}
\item \ctyping[][\WContextTrans{\Context}]{\WTrans{N}}{\WTypeTrans{A}}
\end{itemize}
Which allows to conclude by the \CicApp{} rule.
\end{ProofCase}

\begin{ProofCase}{\CicConv}
The induction gives us
\begin{itemize}
\item \ctyping[][\WContextTrans{\Context}]{\WTrans{M}}{\WTypeTrans{B}}
\item \ctyping[][\WContextTrans{\Context}]{\WTrans{A}}{\WTypeTrans{s}}
\item \convertible[\WContextTrans{\Context}]{\WTrans{A}}{\WTrans{B}}
\end{itemize}
We know that \ctyping[][\WContextTrans{\Context}]{\WTypeTrans{A}}{s} by 
\lemmaRef{lemma:wtrans_sort_involutive}.
We need to check if \convertible{\WTypeTrans{A}}{\WTypeTrans{B}}.
There are four cases:
\begin{itemize}
\item \convertible[\WContextTrans{\Context}]{\app{\El}{\WTrans{A}}}{\app{\El}{\WTrans{B}}}.
\item \convertible[\WContextTrans{\Context}]{\WTrans{A}}{\WTrans{B}}.
\item \convertible[\WContextTrans{\Context}]{\WTrans{A}}{\app{\El}{\WTrans{B}}}.
\item \convertible[\WContextTrans{\Context}]{\app{\El}{\WTrans{A}}}{\WTrans{B}}.
\end{itemize}
The first and second cases are allowed because of the hypothesis and conversion. The third and fourth are 
impossibles because if the terms have those forms, it is because one belongs to \Prop{} and the other 
to \Type{}, and those sorts are not convertible in the source and target theories.
Now is possible to apply \CicConv{} which gives the result.
\end{ProofCase}

\begin{ProofCase}{\CicIdem}
We need to prove
\begin{itemize}
\item $\ctyping[\typing{x}{\WTypeTrans{A}}][\wfContext{\Context}]{x}{\WTypeTrans{A}}$
\end{itemize}
given
\begin{itemize}
\item $\ctyping[][\wfContext{\Context}]{\WTrans{A}}{\WTypeTrans{s}}$
\end{itemize}
By \lemmaRef{lemma:wtrans_sort_involutive} we have 
$\ctyping[][\WContextTrans{\Context}]{\WTypeTrans{A}}{s}$ which allows us to conclude
by \CicIdem{} rule
\end{ProofCase}

\end{Proof}



\noindent------------------------------------------------------------------------------------------------------------------------------------------------------------------------------------------------------------------------------------------------------------------------------------------------------------------------


This change on the propositional sort forces the purity on proofs because:
\begin{center}
    $\TypeTrans{\Prop}
    \equiv
    \app{\El}{\Trans{\Prop}}
    \equiv
    \app{\El}{ (\app{\TypeVal}{\Prop}{(\abs{\,\noBinder}{\ExceptionType}{
                                            (\depArrow{A}{\Prop}{A}) })}) }
    \equiv
    \Prop$
\end{center}



The first step to achieving this goal is to be able to identify when we are constructing
a proposition or a proof, from other terms. This means that we would like a function that
receives terms of the calculus, specifically types, and returns true if the arguments is 
either a proposition or an arity of \Prop{} (the argument constructs a proposition),
and false otherwise. For example:
\begin{enumerate}
\item $\depArrow{A}{\Type}{x}{A}{\app{\Indeq}{A}{x}{x}}$
\item $\depArrow{n}{\Indnat}{P}{\arrow{nat}{\Type}}{\arrow{\app{P}{n}}{\Prop}}$
\item $\app{(\abs{A}{\Type}{A})}{\Prop}$
\item $\depArrow{A}{\Type}{x}{A}{\arrow{\app{\Indeq}{A}{x}{x}}{nat}}$
\item $\Type$
\end{enumerate}
In the previous examples, the function that must be defined should output true for cases one, two and three,
where case one correspond to a proposition, and cases two and three are arities (after reduction in the third
case), and false for case four and five

The previous is captured by the function \propFun{} defined in \figRef{fig:ArToPropDef}.
This function 
detects if a type is an arity
of \Prop{}, a type constructor belonging to \Prop{}, or a proposition. It would be said that
a term is related to \Prop{} if the result of \propFun{} is true and related to \Type{} if the 
result is false. Note that \propFun{} behaves in the expected way in the previous cases.
\begin{figure}
    \begin{tabular}{l@{ }l}
    \propFun[M] = & $\! \left \{ \begin{matrix*}[l]
                    \trueTerm & if M:\Prop\\
                    \trueTerm & \text{if }M\text{ is an arity of }\Prop\\
                    \falseTerm & \text{otherwise}
                    \end{matrix*} \right .$
    \end{tabular}
    \caption{Definition of $\propFun$ function}
    \label{fig:ArToPropDef}
\end{figure}

Now we are able to define the translation (see \figRef{fig:weaklyParExcTransDef}).
As stated before, this translation is a variation of the exceptional one. It behaves in the same way 
as in the case of the type hierarchy but not in the proposition hierarchy. Essentially, the 
translation detects when the te







The idea is to use \propFun{} on the types to guide how the translation should behave. If for a 
given type the result of \propFun{} is true, then we should relax the parametric argument living in \Type{}.
In fact, we should relax the terms whose result of \propFun{} is false (the second case 
in \todo{move to figure} lives in \Type{} but we would like to be parametric over it). 
For this we use \propFun{} in the abstractions, dependent products and applications, which produces branches in
the definition on those points. Those branches allow the translated term 
to be non-parametric on the argument, which enables to 
reason about the impure terms. Because the condition is repeated along the three cases, we define 
\begin{center}
$\weaklypropFun[A,B]\termDef\neg\ \propFun[A] \wedge \propFun[B]$
\end{center}
The use of this function is similar in all the cases. In the abstraction case, for example, 
$\weaklypropFun[A,B]$
is true when $A$ is related to \Type{} but $B$ is in \Prop{}.\\
Is important to note that context translation is parametrized with a boolean instead of a context.
This is due to the fact that the context is used to check a term. The boolean reflects whether the 
translation should be parametric over the terms related to \Type.
In practice, this boolean is equal to the result of apply the \propFun{} over the term on which 
the context is needed.

As usual, to establish the 
soundness of the translation, we need a few fundamental lemmas. It is important to recall that 
the exceptional and parametric exceptional translation are sound.

\begin{figure}
    \centering
    \begin{minipage}{\textwidth}
    \renewcommand{\arraystretch}{1.15}
    \begin{tabular}{l@{$\hspace{0.1em}\termDef$\ \ }l}
    \WParamTrans{\Type} & \abs{A}{\TypeTrans{\Type}}{\TypeTrans{A} \rightarrow \Type} \\
    \WParamTrans{\Prop} & \WParamTransProp \\
    \WParamTrans{x} & $\WTermTrans{x}$ \\
     
    \WParamTrans{\abs{x}{A}{M}} & $\! \! \left \{ \begin{matrix*}[l]
                                  \WParamTransPropAbs & \text{where }\ctyping{M}{B}\text{ and }\weaklypropFun[A,B]\\
                                  \WParamTransNPropAbs & \otherwise
                                  \end{matrix*} \right.$ \\
    \WParamTrans{\app{\M{}}{\N{}}} & $\! \! \left \{ \begin{matrix*}[l]
                                \app{\WParamTrans{M}}{\Trans{N}} & \text{where }\ctyping{M}{(\vardepArrow{A}{B})}
                                \text{ and }\weaklypropFun[A,B]\\
                                \app{\WParamTrans{M}}{\Trans{N}}{\WParamTrans{N}} & \otherwise
                                \end{matrix*} \right.$ \\
    \WParamTrans{\depArrow{x}{A}{B}} & $\! \! \left \{ \begin{matrix*}[l]
                                      \varabs{f}{\depArrow{x}{\TypeTrans{A}}{\TypeTrans{B}}}  
                                        {\depArrow{x}{\TypeTrans{A}}{
                                                   \WParamTypeTrans[\Context,\smalltyping{x}{A}]{B} (\app{f}{x})}} &
                                            \weaklypropFun[A,B] \\
                                      \varabs{f}{\depArrow{x}{\TypeTrans{A}}{\TypeTrans{B}}}  
                                        {\depArrow{x}{\TypeTrans{A}}
                                                  {\WTermTrans{x}}{\WParamTypeTrans{A}\ x}{
                                                   \WParamTypeTrans[\Context,\smalltyping{x}{A}]{B} (\app{f}{x})} 
                                                  } 
                                                        & \otherwise
                                      \end{matrix*} \right.$ \\
    \WParamTypeTrans{A} & \WParamTrans{A} \vspace{2mm}\\\hline
    \rule{0pt}{15pt}\WContextTrans{\EContext} & \EContext \\
    \WContextTrans{\Context, \typing{x}{A}} & $\!\! \left \{ \begin{matrix*}[l] 
    \WContextTrans{\Context}, \typing{x}{\TypeTrans{A}}  & 
                              \text{where }b = \True \text{ and }\neg\ \propFun[A]\\
    \WContextTrans{\Context}, \typing{x}{\TypeTrans{A}}, 
                              \typing{\WTermTrans{x}}{\app{\WParamTypeTrans{A}}{x}} & \otherwise
    \end{matrix*}\right .$
    \end{tabular}
    \end{minipage}
    \caption{Weakly Parametric Exceptional Translation}
    \label{fig:weaklyParExcTransDef}
\end{figure}

\begin{Lemma}[Substitution]
\label{lemma:substitution_lemma}
Given a context \Context and two terms \M{} and \N{} such as \ctyping{\M{}}{B},
\ctyping{\N{}}{A}, we have:
\begin{center}
$
\renewcommand{\arraystretch}{1.15}
\WParamTrans{M\{x:=N\}} \equiv 
\left \{\begin{matrix*}[l]
    \WParamTrans{M}\{x:=\Trans{N}\} & {\normalfont\weaklypropFun[A,B] }\\
    \WParamTrans{M}\{x:=\Trans{N};\ \WTermTrans{x}:=\WParamTrans{N}\} & \otherwise
\end{matrix*}
\right .
$
\end{center}

\end{Lemma}

\begin{Proof} By induction over M over a generalized context.

\begin{ProofCase}{\M{} \termDef{} \Type{}, \M{} \termDef{} \Prop{}}
In both of this cases, whether $x = A$ or $x \neq A$,
applying the substitution over the abstraction does not affect it and the substitution holds.
\end{ProofCase}

\begin{ProofCase}{\M{} \termDef{} \y{}, $y \neq x$}
The equivalence holds trivially.
\end{ProofCase}

\begin{ProofCase}{\M{} \termDef{} $x$}
Because the context has unique identifiers, we can conclude that 
$A\equiv B$. It is also clear that $\weaklypropFun[A,A]\termEq\False$ from definition.
We can conclude then:
\begin{center}
$\WParamTrans{\subst{x}{x}{N}}\equiv\subst{\WParamTrans{x}}{x}{\Trans{N}}{\WTermTrans{x}}{\WParamTrans{N}\}}$
\end{center}

\end{ProofCase}

\begin{ProofCase}{\M{} \termDef{} \abs{y}{C}{M'}}
If $y = x$, then equivalence holds because there is no need to substitute \x{} in \M{} and this is respected
in the translation. Assuming $y \neq x$ we have \mbox{\WParamTrans{\abs{y}{C}{(M'\{x:=N\})}}}. 
By the inversion lemma we know that $B=\depArrow{y}{C}{D}$ for some $C$ and $D$. Note that the translation
works on the previous types and $A$ does not affect it. By unfolding the translation we can 
get either:
\begin{itemize}
\item \abs{y}{\TypeTrans{\subst{C}{x}{N}}}{
                \WParamTrans[\Context,\typing{y}{C}]{\subst{M'}{x}{N}}}
\item \abs{y}{\TypeTrans{\subst{C}{x}{N}}}{
           \WTermTrans{y}}{\app{\WParamTypeTrans{\subst{C}{x}{N}}}{y}}{
                \WParamTrans[\Context,\typing{y}{C}]{\subst{M'}{x}{N}}}
\end{itemize}
Each case requires to do a case analysis over \weaklypropFun[A,B]. Both cases are similar: we show
here the second case because is more general. Note that the translation for $M$ uses other context. This 
is solved by applying \lemmaRef{lemma:ContextInvariance} on the $N$ context so the induction hypothesis
can be applied with the extended context. By applying the induction hypothesis plus exceptional substitution
(defining $C'\termDef\subst{\TypeTrans{C}}{x}{\Trans{N}}$) we have:
\begin{itemize}
\item \abs{y}{C'}{
           \WTermTrans{y}}{\app{(\subst{\WParamTypeTrans{C}}{x}{N})}{y}}{
                \subst{\WParamTrans[\Context,\typing{y}{C}]{M'}}{x}{N}}
\item \abs{y}{C'}{
           \WTermTrans{y}}{\app{(\subst{\WParamTypeTrans{C}}{x}{N}{\WTermTrans{x}}{\WParamTrans{N}})}{y}}{
                \subst{\WParamTrans[\Context,\typing{y}{C}]{M'}}{x}{N}{\WTermTrans{x}}{\WParamTrans{N}}}
\end{itemize}
It is clear in each case that they are equal to:
\begin{itemize}
\item \subst{(\abs{y}{\TypeTrans{C}}{
           \WTermTrans{y}}{\app{\WParamTypeTrans{C}}{y}}{
                \WParamTrans[\Context,\typing{y}{C}]{M'}})}{x}{\Trans{N}}
\item \subst{(\abs{y}{\TypeTrans{C}}{
           \WTermTrans{y}}{\app{\WParamTypeTrans{C}}{y}}{
                \WParamTrans[\Context,\typing{y}{C}]{M'}})}{x}{\Trans{N}}{\WTermTrans{x}}{\WParamTrans{N}}
\end{itemize}
The last one is due to $\WTermTrans{x}$ being a fresh identifier
\end{ProofCase}

\begin{ProofCase}{\M{} \termDef{} \depArrow{y}{A}{B}}
Similar to the previous case. If $y = x$, then equivalence holds. If $y \neq x$, unfolding the translation, 
applying the induction hypothesis and folding the substitution gives the result.
\end{ProofCase}

\begin{ProofCase}{\M{} \termDef{} \app{M'}{N'}}
Here we need to use the inversion lemma to establish the form of $B$. Once established it is possible to unfold
the translation and apply both the induction hypothesis and the exceptional substitution.
\newcommand{\substEnv}[0]{\ensuremath{\{x\termDef{} \Trans{N}\}}}
\newcommand{\substEnvv}[0]{\ensuremath{\{x\termDef{} N\,;\ \WTermTrans{x}\termDef\WParamTrans{N}\}}}
\begin{itemize}[noitemsep]
    
    \item \app{L_1}{H}
    \item \app{L_2}{H}
    \item \app{L_1}{H}{K_1}
    \item \app{L_2}{H}{K_1}
    \item \app{L_1}{H}{K_2}
    \item \app{L_2}{H}{K_2}
\end{itemize}
Where $L_1\termDef\WParamTrans{M'}\substEnv, L_2\termDef\WParamTrans{M'}\substEnvv,
K_1\termDef\WParamTrans{N'}\substEnv, K_2\termDef\WParamTrans{N'}\substEnvv, H\termDef\Trans{N'}\substEnv$.\\
If all the subindexes of the resulting term are 1, then it is equivalent to
$(\WParamTrans{\app{M'}{N'}}\substEnv)$. If not, then it is equivalent 
to $(\WParamTrans{\app{M'}{N'}}\substEnvv)$. The last result
is due to the fact that substitute for a free variable ($\WTermTrans{x}$) in a term does not affect it. 
\end{ProofCase}
\end{Proof}

\begin{Lemma}[Conversion]
\label{lemma:conversion}
Given two terms \ctyping{\M{}}{A}, \ctyping{\N{}}{B}, if \convertible{\M{}}{\N{}} then
\convertible{\WParamTrans{\M{}}}{\WParamTrans{\N{}}}
\end{Lemma}

\begin{Proof}
By induction on the conversion rule.
\begin{ProofCase}{$\beta$-reduction}
We have $\M{}\termDef\app{(\abs{x}{A}{M'})}{N'}$ and $\N{}\termDef\M{}'\{x\termDef \N{}'\}$. The 
translation over \M{} could be \mbox{$\app{(\WParamTransPropAbs[x][A][M])}{\Trans{N'}}$} or 
\mbox{$\app{(\WParamTransNPropAbs[x][A][M])}{\Trans{N'}}{\WParamTrans{N'}}$}. 
Both terms belong to the language and it is possible to apply $\beta$-reductions; one in the former and 
two in the latter. Applying \lemmaRef{lemma:substitution_lemma} over N establishes the equivalence.
\end{ProofCase}

\begin{ProofCase}{$\eta$-expansion}
The translation maps functions to functions. It is necessary to establish the following equivalence
\begin{center}
$\WParamTrans{\abs{x}{A}{\app{M}{x}}} \termDef \left \{
\begin{matrix*}[l]
    \app{\WParamTransPropAbs}{x} \\
    \app{\WParamTransNPropAbs}{x}{\WTermTrans{x}}
\end{matrix*} \right \} \equiv \WParamTrans{M}$    
\end{center}
In both cases the equivalence holds because those terms belong to the language and $\eta$-expansion 
can be applied.
\end{ProofCase}

\begin{ProofCase}{application}
We have $\M\termDef(\app{\M_1}{\N_1})$ and $\N\termDef(\app{\M_2}{\N_2})$. There are two cases:
\begin{itemize}
    \item $\M_1\equiv\M_2$ and $\N_1=\N_2$
    \item $\M_1=\M_2$ and $\N_1\equiv\N_2$
\end{itemize}
In each case it is enough to unfold the translation and use the inductive hypothesis. 
Depending on the case, it might need
the fact that if \convertible{N}{M} then \convertible{\Trans{N}}{\Trans{M}}
\end{ProofCase}
\end{Proof}

The following lemma demonstrate a simple result of the typing judgments.
\begin{Lemma}[Well-formed of context]
Given $\Context,M,A$ such as \ctyping{M}{A}, then \wfContext{\Context} 
\end{Lemma}

\begin{Proof}
This is a straightforward induction on the typing derivation. The well-formed
of the context is proved recursively at each typing judgment.
\end{Proof}

Here we state a classic lemma for the lambda calculus.
\begin{Lemma}[Context permutation]
Given $\Context,M,A$ such as \ctyping{M}{A}, then given a 
context $\varContext$ that is a permutation of \Context{}, then \ctyping[][\varContext]{M}{A}.
\end{Lemma}

\begin{WarningProof}
\textsc{Todo}. By induction on the typing derivation. The \CicWeak case does not go through.
\end{WarningProof}

The following lemma establish that it is possible to change the context for a 
bigger one. It is necessary to keep in mind that the bindings declared 
in a context are unique.
\begin{Lemma}[Context invariance]
\label{lemma:ContextInvariance}
Given $\Context,M,A$ such as \ctyping{M}{A}, then given a 
context $\varContext$ such as 
$\Context \subseteq  \varContext$ we have \ctyping[][\varContext]{M}{A}
\end{Lemma}

\begin{Proof}
By induction on the typing judgment 
\begin{SubProofCase}{\CicType}
We need to prove $\ctyping[][\varContext]{\Type}{\Type[j]}$. 
This is straight forward becuase \varContext{} is well-formed.
\end{SubProofCase}

\begin{SubProofCase}{\CicProp}
Similar to the previous case.
\end{SubProofCase}

\begin{SubProofCase}{\CicWeak}
We have to prove $\ctyping[][\varContext]{M}{B}$ with $\Context \subseteq \varContext$. We know
that the variable $a$ belongs to \varContext{} so we define
$\varContext'\termDef\varContext,\typing{a}{A}$.
and apply the permutation lemma. We have to prove now 
$\ctyping[\typing{a}{A}][\varContext']{M}{B}$. By applying the induction hypothesis with 
$\varContext'$ we get $\ctyping[][\varContext']{M}{B}$ and $\ctyping[][\varContext']{A}{s}$,
which allows us to apply the \CicWeak{} judgment.
\end{SubProofCase}

\begin{SubProofCase}{\CicTypeProd, \CicPropProd, \CicImpred, (\CicAbs)}
In all these cases the proof is similar. We have a context \varContext{} that is used to 
type the product (lambda).
Using the induction hypothesis with \varContext{} for the codomain (product) and with 
$\varContext,\typing{x}{A}$ for the domain (body),
we get the hypothesis to conclude with the respective 
typing judgment $\ctyping[][\varContext]{\depArrow{x}{A}{B}}{s}$ 
($\ctyping[][\varContext]{\abs{x}{A}{M}}{\depArrow{x}{A}{B}}$)
\end{SubProofCase}

\begin{SubProofCase}{\CicConv, \CicApp}
These cases are straightforward because the context does not change.
\end{SubProofCase}
\end{Proof}

\begin{Lemma}[Context weakening]
\label{lemma:ContextWeakening}
Given $\Context,M,A$ such as \ctyping{M}{A}, then given a 
context $\FreeVar{M}\varContext\subseteq\Context$ such that
$\Context' \subseteq\FreeVar{M}$ and $\varContext\termDef\Context-\Context'$ we have 
$\ctyping[][\varContext]{M}{A}$.
\end{Lemma}
\begin{Proof}
This proof is similar to \lemmaRef{lemma:ContextInvariance}
\end{Proof}

\begin{Theorem}[Typing Soundness]
\label{theorem:typingSoundness}
If \ctyping{M}{A} then 
\ctyping[][\WContextTrans[b]{\Context}]{\WParamTrans{M}}{\app{\WParamTypeTrans{A}}{\Trans{M}}} with 
$b=\propFun[A]$.
\end{Theorem}

\begin{Proof}
By induction over the typing derivation.
\begin{ProofCase}{\CicType}
The typing relation to prove is the following:
\begin{center}
\ctyping[][\WContextTrans[\False]{\Context}]{\WParamTrans{\Type}}{\app{\WParamTypeTrans{\Type[j]}}{\Trans{\Type}}}
\termEq{}
\ctyping[][\WContextTrans[\False]{\Context}]{%
    \abs{A}{\TypeTrans{\Type}}{\arrow{\TypeTrans{A}}{\Type}}}{%
    \app{(\abs{A}{\TypeTrans{\Type[j]}}{\arrow{\TypeTrans{A}}}}{\Type[j])}{\Trans{\Type}}
    }
\end{center}
The type of the abstraction is well-typed because $\Trans{\Type}:\type$, $\type:\type[j]$ and 
$\type[j]\equiv\TypeTrans{\Type[j]}$. Applying \CicConv with the $\beta$-reduced type we get:
\begin{center}
\ctyping[][\WContextTrans[\False]{\Context}]{%
    \abs{A}{\TypeTrans{\Type}}{\arrow{\TypeTrans{A}}{\Type}}}{%
    \arrow{\TypeTrans{\Type}}{\Type[j]}
    }
\end{center}
which holds by \CicAbs.
\end{ProofCase}

\begin{ProofCase}{\CicProp}
Similar to the previous case:
\begin{center}
\ctyping[][\WContextTrans[\True]{\Context}]{\WParamTrans{\Prop}}{\app{\WParamTypeTrans{\Type[j]}}{\Trans{\Prop}}}
\termEq{} 
\ctyping[][\WContextTrans[\True]{\Context}]{%
    \abs{A}{\TypeTrans{\Prop}}{\arrow{\TypeTrans{A}}{\Prop}}}{%
    \app{(\abs{A}{\TypeTrans{\Type}}{\arrow{\TypeTrans{A}}}}{\Type)}{\Trans{\Prop}}
    }
\end{center}
With $\Trans{\Prop{}}:\type$, $\type\equiv\TypeTrans{\Type}$. Applying a $\beta$-reduction on the type gives:
\begin{center}
\ctyping[][\WContextTrans[\True]{\Context}]{%
    \abs{A}{\TypeTrans{\Prop}}{\arrow{\TypeTrans{A}}{\Prop}}}{%
    \arrow{\TypeTrans{\Prop}}{\Type}
    }
\end{center}
which holds by \CicAbs.
\end{ProofCase}

\begin{WarningProofCase}{\CicWeak}
\textsc{Disclaimer: The rules of the calculus does not allows this proof due to the weakening is
only allowed on sorts. check}

\noindent We need to prove:
\begin{center}
\ctyping[][\WContextTrans{\Context,\typing{x}{A}}]{%
           \WParamTrans[\Context,\typing{x}{A}]{M}}{\app{\WParamTypeTrans[\Context,\typing{x}{A}]{B}}{\Trans{M}}}    
\end{center}

This case is the weakening. We know by induction that \WParamTypeTrans{A} is an abstraction of type
\arrow{\TypeTrans{A}}{s} and that
$\ctyping[\WContextTrans{\Context}][]{\WParamTrans{M}}{\app{\WParamTypeTrans{B}}}{\Trans{M}}$ 
with $b=\weaklypropFun[B]$. It only remains to do a case analysis over $b$ and $\propFun[A]$,
which can extend context with
\begin{center}
$\typing{x}{\TypeTrans{A}}$ 
\hspace{2em}
or
\hspace{2em}
\mbox{$\typing{x}{\TypeTrans{A}},\typing{\WTermTrans{x}}{\app{\WParamTrans{A}}{x}}$}    
\end{center}
Each one is well-founded so is possible to apply the weakening rule once and twice, respectively,
to establish the result.
\end{WarningProofCase}

\begin{ProofCase}{\CicPropProd}
We need to prove:
\begin{center}
$\ctyping[][\WContextTrans[\False]{\Context}]{%
           \WParamTrans{\depArrow{x}{A}{B}}}{%
           \app{\WParamTypeTrans{\Type}}{\Trans{\depArrow{x}{A}{B}}}}
$
\end{center}
Using the hypothesis we get $\propFun[\depArrow{x}{A}{B}]\termEq\False$, which gives:
\begin{center}
$\ctyping[][\WContextTrans[\False]{\Context}]{%
           \varabs{f}{\depArrow{x}{\TypeTrans{A}}{\TypeTrans{B}}}{%
                  \depArrow{x}{\TypeTrans{A}}{\WTermTrans{x}}{\app{\WParamTypeTrans{A}}{x}}{
                            \app{\WParamTypeTrans[\Context,\typing{x}{A}]{B}}{(\app{f}{x})}}}}{%
           \arrow{\TypeTrans{\depArrow{x}{A}{B}}}{\Type}}
$    
\end{center}
Applying \CicAbs{} and defining
$\varContext\termDef\WContextTrans[\False]{\Context},\typing{f}{\depArrow{x}{\TypeTrans{A}}{\TypeTrans{B}}}$
we have to show:
\begin{itemize}
\item $\ctyping[][\WContextTrans[\False]{\Context}]{\depArrow{x}{\TypeTrans{A}}{\TypeTrans{B}}}{\Type}$
\item $\ctyping[][\varContext]{\depArrow{x}{\TypeTrans{A}}{\WTermTrans{x}}{\app{\WParamTypeTrans{A}}{x}}{
                                       \app{\WParamTypeTrans[\Context,\typing{x}{A}]{B}}{(\app{f}{x})}}}{
                             \Type}$
\end{itemize}
The first proof follows by the soundness of the exceptional translation, the fact that 
$\TypeTrans{\Context} \subseteq \WContextTrans[\False]{\Context}$ and by 
\lemmaRef{lemma:ContextInvariance}. For the second one, it is necessary to apply
\CicPropProd{}. Now we have to prove two another cases:
\begin{itemize}
\item $\ctyping[][\varContext]{\TypeTrans{A}}{\Prop}$
\item $\ctyping[][\varContext',\typing{x}{\TypeTrans{A}}]{
                  \depArrow{\WTermTrans{x}}{\app{\WParamTypeTrans{A}}{x}}{
                            \app{\WParamTypeTrans[\Context,\typing{x}{A}]{B}}{(\app{f}{x})}}}{
                  \Type}$
\end{itemize}
Same as before, the first property holds by the soundness of the exceptional translation and 
context invariance. Applying \CicTypeProd{} for the second case
we have to prove:
\begin{itemize}
\item $\ctyping[][\varContext,\typing{x}{\TypeTrans{A}}]{\app{\WParamTypeTrans{A}}{x}}{\Prop}$
\item $\ctyping[][\varContext,\typing{x}{\TypeTrans{A}},\typing{\WTermTrans{x}}{\app{\WParamTypeTrans{A}}{x}}]{
                  \app{\WParamTypeTrans[\Context,\typing{x}{A}]{B}}{(\app{f}{x})}}{
                  \Type}$
\end{itemize}
For the first case, we apply the induction hypothesis on $\ctyping{A}{\Prop}$ which gives
$\ctyping[\WContextTrans[\True]{\Context}][]{\WParamTypeTrans{A}}{\arrow{\TypeTrans{A}}{\Prop}}$.
It is important to note that
the context is $\WContextTrans[\True]{\Context}$ instead of $\WContextTrans[\False]{\Context}$. We also 
know that $\WContextTrans[\True]{\Context} \subseteq \WContextTrans[\False]{\Context}$ due to the
definition of the translation. With this in mind, we apply \lemmaRef{lemma:ContextInvariance}
to derive 
$\ctyping[\varContext,\typing{x}{\TypeTrans{A}}][]{\WParamTypeTrans{A}}{\arrow{\TypeTrans{A}}{\Prop}}$
and applying the \CicApp{} judgment allows us to conclude the first case. The remaining case is similar to the 
previous: apply the induction hypothesis to $\ctyping[\typing{x}{A}]{B}{\Type}$ gives 
$\ctyping[][\WContextTrans[\False],\typing{x}{\TypeTrans{A}},\typing{\WTermTrans{x}}{\app{\WParamTypeTrans{A}}{x}}]{
        \WParamTypeTrans[\Context,\typing{x}{A}]{B}}{\arrow{\TypeTrans{B}}{\Type}}$.
Applying \lemmaRef{lemma:ContextInvariance} to extend the context, and applying the \CicApp{} judgment two times 
concludes the proofing for this case.
\end{ProofCase}

\begin{ProofCase}{\CicTypeProd}
Similar to the previous case. With $\propFun[\depArrow{x}{A}{B}]\termEq\False$ we need to prove:
\begin{center}
$\ctyping[][\WContextTrans[\False]{\Context}]{%
           \varabs{f}{\depArrow{x}{\TypeTrans{A}}{\TypeTrans{B}}}{%
                  \depArrow{x}{\TypeTrans{A}}{\WTermTrans{x}}{\app{\WParamTypeTrans{A}}{x}}{
                            \app{\WParamTypeTrans[\Context,\typing{x}{A}]{B}}{(\app{f}{x})}}}}{%
           \arrow{\TypeTrans{\depArrow{x}{A}{B}}}{\Type}}
$
\end{center}
The proof goes in a similar way to the \CicPropProd{} case, with the only difference that we need to prove
$\ctyping[][\varContext]{\TypeTrans{A}}{\Type}$
instead of 
$\ctyping[][\varContext]{\TypeTrans{A}}{\Prop}$.
This is done by applying the induction hypothesis over it, which gives 
$\ctyping[\WContextTrans[\False]{\Context}][]{\WParamTypeTrans{A}}{\arrow{\TypeTrans{A}}{\Type}}$. Note
that the translated context is the same along the proof. The rest remains the same.
\end{ProofCase}

\begin{ProofCase}{\CicImpred}
It is necessary to perform a case analysis over the sort of the argument.
\begin{SubProofCase}{$s\termDef\Prop$}
This case is equal to the \CicTypeProd case with \Type{} and \WContextTrans[\False]{\Context}
replaced by \Prop{} and \WContextTrans[\True]{\Context}, respectively.  
\end{SubProofCase}

\begin{SubProofCase}{$s\termDef\Type$}
In this case we need to prove:
\begin{center}
$\ctyping[][\WContextTrans[\True]{\Context}]{%
           \varabs{f}{\depArrow{x}{\TypeTrans{A}}{\TypeTrans{B}}}{%
                  \depArrow{x}{\TypeTrans{A}}{
                            \app{\WParamTypeTrans[\Context,\typing{x}{A}]{B}}{(\app{f}{x})}}}}{%
           \arrow{\TypeTrans{\depArrow{x}{A}{B}}}{\Prop}}
$
\end{center}
By \CicAbs{} and defining
$\varContext\termDef\WContextTrans[\True]{\Context},\typing{f}{\depArrow{x}{\TypeTrans{A}}{\TypeTrans{B}}}$
we get:
\begin{itemize}
\item $\ctyping[][\WContextTrans[\True]{\Context}]{\depArrow{x}{\TypeTrans{A}}{\TypeTrans{B}}}{\Prop}$
\item $\ctyping[][\varContext]{\depArrow{x}{\TypeTrans{A}}{
                                       \app{\WParamTypeTrans[\Context,\typing{x}{A}]{B}}{(\app{f}{x})}}}{
                             \Prop}$
\end{itemize}
In the first proof, we have 
\ctyping[][\TypeTrans{\Context}]{\depArrow{x}{\TypeTrans{A}}{\TypeTrans{B}}}{\Prop} by the 
soundness of the exceptional translation. We also now that
$\TypeTrans{\Context} \subseteq \WContextTrans[\True]{\Context}$. This two hypothesis allow us to conclude 
the subproof by \lemmaRef{lemma:ContextInvariance}. 
In the second case, it is necessary to apply \CicAbs{}, which raises another two cases:
\begin{itemize}
\item $\ctyping[][\varContext]{\TypeTrans{A}}{\Type}$
\item $\ctyping[][\varContext,\typing{x}{\TypeTrans{A}}]{
                                       \app{\WParamTypeTrans[\Context,\typing{x}{A}]{B}}{(\app{f}{x})}}{
                             \Prop}$
\end{itemize}
The first case is proved by the soundness of exceptional translation and context invariance. For the second one 
we apply the induction hypothesis over \ctyping[\typing{x}{A}]{B}{\Prop} which gives:
\begin{center}
$\ctyping[\typing{x}{\TypeTrans{A}}]{\WParamTypeTrans[\Context,\typing{x}{A}]{B}}{\arrow{\TypeTrans{B}}{\Prop}}$
\end{center}
We conclude by applying \CicAbs{} twice.
\end{SubProofCase}
\end{ProofCase}

\begin{ProofCase}{\CicConv}
By induction hypothesis we have:
\begin{itemize}
    \item \ctyping[][\WContextTrans{\Context}]{\WParamTrans{M}}{\app{\WParamTypeTrans{B}}{\Trans{M}}}.
    \item \ctyping[][\WContextTrans{\Context}]{\WParamTrans{A}}{\arrow{\TypeTrans{A}}{s}}.
    \item \convertible{\WParamTrans{A}}{\WParamTrans{B}} by \lemmaRef{lemma:conversion}.
\end{itemize}
It remains to establish
\ctyping[][\WContextTrans{\Context}]{\WParamTrans{M}}{\app{\WParamTypeTrans{A}}{\Trans{M}}} which follows by 
\CicConv.
\end{ProofCase}

\begin{ProofCase}{\CicAbs}
In this case is necessary to perform a case analysis over the result of \weaklypropFun{}.
\begin{SubProofCase}{$\weaklypropFun[A,B] = \True$}
For this case we need to prove:
\begin{center}
$\ctyping[][\WContextTrans{\Context}]{\abs{x}{\TypeTrans{A}}{\WParamTrans{M}}}{
        \app{(\varabs{f}{\depArrow{x}{\TypeTrans{A}}{\TypeTrans{B}}}{
                       \depArrow{x}{\TypeTrans{A}}{
                         \app{\WParamTypeTrans[\Context,\typing{x}{A}]{B}}{
                              (\app{f}{x})}}})}}{
             (\abs{x}{\TypeTrans{A}}{\Trans{M}})}$
\end{center}
Note that the application on the type is valid because the lambda has type 
\depArrow{x}{\TypeTrans{A}}{\TypeTrans{B}}. By \CicConv{} we need to prove:
\begin{center}
$\ctyping[][\WContextTrans{\Context}]{\abs{x}{\TypeTrans{A}}{\WParamTrans{M}}}{
        \depArrow{x}{\TypeTrans{A}}{(
                    \app{\WParamTypeTrans[\Context,\typing{x}{A}]{B}}{(\subst{\Trans{M}}{x}{x})})}}$
\end{center}
Also, in this setting we have $M\termEq\subst{\Trans{M}}{x}{x}$ Applying \CicAbs{} raises two cases:
\begin{itemize}
\item $\ctyping[][\WContextTrans{\Context}]{
               \depArrow{x}{\TypeTrans{A}}{(
                    \app{\WParamTypeTrans[\Context,\typing{x}{A}]{B}}{\Trans{M}})}}{
                s}$
\item $\ctyping[][\WContextTrans{\Context},\typing{x}{\TypeTrans{A}}]{\WParamTrans[\Context,\typing{x}{A}]{M}}{
                \app{\WParamTypeTrans[\Context,\typing{x}{A}]{B}}{\Trans{M}}}$
\end{itemize}
For the first case we apply the induction hypothesis over 
$\ctyping{\depArrow{x}{A}{B}}{s}$ which gives:
\begin{center}
$\ctyping[][\WContextTrans{\Context}]{
            \varabs{f}{\depArrow{x}{\TypeTrans{A}}{\TypeTrans{B}}}{
                       \depArrow{x}{\TypeTrans{A}}{
                         \app{\WParamTypeTrans[\Context,\typing{x}{A}]{B}}{
                              (\app{f}{x})}}}}{
              \arrow{\depArrow{x}{\TypeTrans{A}}{\TypeTrans{B}}}{s}}$
\end{center}
When the previous term is applied to $\abs{x}{\TypeTrans{A}}{\Trans{M}}$, which exists by 
the soundness of exceptional translation over the hypotheses and applying \CicAbs{} over them,
and reduced, it allows us to prove the desired term. The application is valid by \CicAbs{}. 
For the second case we apply the induction hypothesis over \ctyping[\typing{x}{A}]{M}{B} gives:
\begin{center}
$\ctyping[][\WContextTrans{\Context,\typing{x}{A}}]{
            \WParamTrans[\Context,\typing{x}{A}]{M}}{
            \app{\WParamTypeTrans[\Context,\typing{x}{A}]{B}}{\Trans{M}}}$
\end{center}
Also, we have
\begin{center}
$\FreeVar{M}\subseteq\WContextTrans{\Context},\typing{x}{A}\subseteq\WContextTrans{\Context,\typing{x}{A}}$
\end{center}
for any value of $b$, so we can apply \lemmaRef{lemma:ContextWeakening}
and establish this case.
\end{SubProofCase}

\begin{SubProofCase}{$\weaklypropFun[A,B] = \False$}
In the same style as before, we prove the reduced judgment:
\begin{center}
$\ctyping[][\WContextTrans{\Context}]{\abs{x}{\TypeTrans{A}}{
                                          \WTermTrans{x}}{\app{\WParamTypeTrans{A}}{x}}{
                                          \WParamTrans{M}}}{
        \depArrow{x}{\TypeTrans{A}}{\WTermTrans{x}}{\app{\WParamTypeTrans{A}}{x}}{(
                    \app{\WParamTypeTrans[\Context,\typing{x}{A}]{B}}{\Trans{M}})}}$
\end{center}
As before, we apply \CicAbs{}:
\begin{itemize}
\item $\ctyping[][\WContextTrans{\Context}]{
               \depArrow{x}{\TypeTrans{A}}{\WTermTrans{x}}{\app{\WParamTypeTrans{A}}{x}}{(
                    \app{\WParamTypeTrans[\Context,\typing{x}{A}]{B}}{\Trans{M}})}}{
                s}$
\item $\ctyping[][\WContextTrans{\Context},\typing{x}{\TypeTrans{A}}]{
                  \varabs{\WTermTrans{x}}{\app{\WParamTypeTrans{A}}{x}}{
                            \WParamTrans[\Context,\typing{x}{A}]{M}}}{
                \enDepArrow{\WTermTrans{x}}{\app{\WParamTypeTrans{A}}{x}}{
                            \app{\WParamTypeTrans[\Context,\typing{x}{A}]{B}}{\Trans{M}}}}$
\end{itemize}
The first case is similar to previous cases: construct the lambda, apply the induction hypothesis
on the type hypothesis, apply the constructed lambda over the previous term and conclude. 
For the second case, is necessary to apply \CicAbs{}:
\begin{itemize}
\item $\ctyping[][\WContextTrans{\Context}]{
               \enDepArrow{\WTermTrans{x}}{\app{\WParamTypeTrans{A}}{x}}{(
                    \app{\WParamTypeTrans[\Context,\typing{x}{A}]{B}}{\Trans{M}})}}{
                s}$
\item $\ctyping[][\WContextTrans{\Context},
                  \typing{x}{\TypeTrans{A}},\typing{\WTermTrans{x}}{\WParamTypeTrans{A}}]{
                \WParamTrans[\Context,\typing{x}{A}]{M}}{
                \app{\WParamTypeTrans[\Context,\typing{x}{A}]{B}}{\Trans{M}}}$
\end{itemize}
In this case we apply the induction hypothesis over $\ctyping[\typing{x}{A}]{M}{B}$ which gives:
\begin{center}
$\ctyping[][\WContextTrans{\Context,\typing{x}{A}}]{
                \WParamTrans[\Context,\typing{x}{A}]{M}}{
                \app{\WParamTypeTrans[\Context,\typing{x}{A}]{B}}{\Trans{M}}}$
\end{center}
Due to the value of the weakly value we have 
$\WContextTrans{\Context,\typing{x}{A}}
\termEq
\WContextTrans{\Context},\typing{x}{\TypeTrans{A}},\typing{\WTermTrans{x}}{\WParamTypeTrans{A}}$,
which concludes this case.
\end{SubProofCase}
\end{ProofCase}

\begin{ProofCase}{\CicApp}
We proceed by performing a case analysis over \weaklypropFun{}
\begin{SubProofCase}{$\weaklypropFun[A,B] = \True$}
We have to prove:
\begin{center}
$\ctyping[\WContextTrans{\Context}][]{\WParamTrans{\app{M}{N}}}{
          \app{\WParamTypeTrans{\subst{B}{x}{N}}}{(\Trans{\app{M}{N}})}}$
\end{center}
Expanding the definitions and applying \lemmaRef{lemma:substitution_lemma}, we get:
\begin{center}
$\ctyping[\WContextTrans{\Context}][]{
         \app{\WParamTrans{M}}{\Trans{N}}}{
         \app{\subst{\WParamTypeTrans{B}}{x}{\Trans{N}}}{
              (\app{\Trans{M}}{\Trans{N}})}}$
\end{center}
Applying the induction hypothesis over M and the soundness of the exceptional translation over N gives:
\begin{itemize}
\item $\ctyping[\WContextTrans{\Context}][]{\WParamTrans{M}}{
                \depArrow{x}{\TypeTrans{A}}{(
                          \app{\WParamTypeTrans[\Context,\typing{x}{A}]{B}}{(\app{\Trans{M}}{x})})}}$
\item $\ctyping[\WContextTrans{\Context}][]{\Trans{N}}{\TypeTrans{A}}$ (plus context invariance)
\end{itemize}
Applying \CicApp{} yields:
\begin{center}
$\ctyping[\WContextTrans{\Context}][]{\WParamTrans{\app{M}{N}}}{
          \subst{(\app{\WParamTypeTrans[\Context,\typing{x}{A}]{B}}{(\app{\Trans{M}}{x})})}{x}{\Trans{N}}}$
\end{center}
Noting that $\subst{\Trans{M}}{x}{\Trans{N}}\termEq\Trans{M}$ we have:
\begin{center}
$\ctyping[\WContextTrans{\Context}][]{
         \app{\WParamTrans{M}}{\Trans{N}}}{
         \app{\subst{\WParamTypeTrans{B}}{x}{\Trans{N}}}{
              (\app{\Trans{M}}{\Trans{N}})}}$
\end{center}
\end{SubProofCase}

\begin{SubProofCase}{$\weaklypropFun[A,B] = \False$}
Same as before, expanding the definitions and applying \lemmaRef{lemma:substitution_lemma} we have to prove:
\begin{center}
\ctyping[\WContextTrans{\Context}][]{
         \app{\WParamTrans{M}}{\Trans{N}}{\WParamTrans{N}}}{
         \app{\subst{\WParamTypeTrans{B}}{x}{\Trans{N}}{\WTermTrans{x}}{\WParamTrans{N}}}{
              (\app{\Trans{M}}{\Trans{N}})}}
\end{center}
Applying the induction hypothesis and the soundness of the exceptional translation yields:
\begin{itemize}
\item $\ctyping[\WContextTrans{\Context}][]{\WParamTrans{M}}{
                \depArrow{x}{\TypeTrans{A}}{\WTermTrans{x}}{\app{\WParamTypeTrans{A}}{x}}{(
                          \app{\WParamTypeTrans[\Context,\typing{x}{A}]{B}}{(\app{\Trans{M}}{x})})}}$
\item $\ctyping[\WContextTrans{\Context}][]{\Trans{N}}{\TypeTrans{A}}$ (plus context invariance)
\item $\ctyping[\WContextTrans{\Context}][]{\WParamTrans{N}}{\app{\WParamTypeTrans{A}}{\Trans{N}}}$
\end{itemize}
\end{SubProofCase}
We can apply \CicApp{} with the first and second hypothesis to get:
\begin{center}
$\ctyping[\WContextTrans{\Context}][]{
         \app{\WParamTrans{M}}{\Trans{N}}}{
              \enDepArrow{\WTermTrans{x}}{\app{\WParamTypeTrans{A}}{\Trans{N}}}{
                         \app{\subst{\WParamTypeTrans{B}}{x}{\Trans{N}}}}{
              (\app{\Trans{M}}{\Trans{N}})}}$
\end{center}
and apply this with the third hypothesis:
\begin{center}
$\ctyping[\WContextTrans{\Context}][]{
         \app{\WParamTrans{M}}{\Trans{N}}}{
              \enDepArrow{\WTermTrans{x}}{\app{\WParamTypeTrans{A}}{\Trans{N}}}{
                         \app{\subst{\WParamTypeTrans{B}}{x}{\Trans{N}}{\WTermTrans{x}}{\WParamTrans{N}}}}{
              (\subst{\app{\Trans{M}}{\Trans{N}}}{\WTermTrans{x}}{\WParamTrans{N}})}}$
\end{center}
Due to $\WTermTrans{x}\in\FreeVar{\app{\Trans{M}}{\Trans{N}}}$ we have that 
$\subst{\app{\Trans{M}}{\Trans{N}}}{\WTermTrans{x}}{\WParamTrans{N}}\termEq\app{\Trans{M}}{\Trans{N}}$
so the judgment holds.
\end{ProofCase}

\noindent The rest of the cases are trivial.
\end{Proof}

\section{Translation of Inductives}

Up to this point, we have considered \CCw{}, which is a system similar to \CIC{} without inductives. We now 
add the following elements to \CCw{}: inductives in the style of \defRef{def:InductiveIntro},
elimination of inductive in the style of \defRef{def:InductiveElim} and the related typing judgment.
This addition forces a modification in the translation to take into account the inductives, 
both the definitions and the appearance in terms.

As stated in \cite{tabareau:FailureIsNotAnOption}, nested inductive types cannot be translated in the 
exceptional layer. Nevertheless,
most of nested inductive types can be represented isomorphically as mutual inductives ones,
which do admit a translation.

\todo{This should be moved to the background part when presenting the translations}
Taking the above into account, the translation of inductives works by defining new ones. One for 
the exceptional layer and another for the weakly parametric that uses the previous one.
Due to the change of syntax, we present how both the exceptional and the weakly parametric exceptional translations
work on inductives.
\begin{Definition}
\label{def:TransInductiveDef}
Let be $\IndIntro[k][\varContext_\IndD][\varContext_C]$ a valid inductive block
definition, as given in \defRef{def:InductiveDef}. 
We define the exceptional translation 
$\IndIntro[k][\varContext_{\ConstructorTrans{\IndD}}][\varContext_{\ConstructorTrans{C}}]$
of the previous block where:
\newcommand{\ArgTrans}[1]{\ensuremath{[\![#1]\!]_{\text{\mathtexttt{Ind}}}}}
\begin{center}
$\varContext_{\ConstructorTrans{\IndD}} 
=
\{\ConstructorTrans{\IndD_i}: \ArgTrans{T} \ |\ \typing{\IndD_i}{T} \in \varContext_\IndD\}
$ \\
$\varContext_{\ConstructorTrans{C}} 
=
\{\ConstructorTrans{c_i}: \ArgTrans{T} \ |\ \typing{c_i}{T} \in \varContext_C\}
$
\end{center}
with $\ArgTrans{\cdot}$ defined as below (note that it is locally redefined what it means the type
translation of inductives)

\begin{tabular}{l@{$\hspace{0.1em}\termDef$\ \ }l}
$\TypeTrans{\app{\IndD}{A_1}{\dots}{\A_{i_\IndD}}}$ & $\app{\ConstructorTrans{\IndD}}{\TypeTrans{A_1}}{
                                                                                      \dots}{
                                                                                      \TypeTrans{A_{i_\IndD}}}$\\
$\ArgTrans{\depArrow{x}{A}{B}}$ & $\depArrow{x}{\TypeTrans{A}}{\ArgTrans{B}}$ \\
$\ArgTrans{A}$ & $\TypeTrans{A}$
\end{tabular}

%\begin{itemize}
%\item parameters $\WIndArgTrans[\cdot]{\typing{p_1}{P_1},\dots,\typing{p_n}{P_n}}$
%\item indices 
%      $\WIndArgTrans[\Context']{\typing{i_1}{I_1},\dots,\typing{i_m}{I_m}},
%       x:\app{\ConstructorTrans{\IndD}}{p_1}{\dots}{p_n}{i_1}{\dots}{i_m}$
%\item constructors \newline
%  $\WTermTrans[1]{c}:\Pi\WIndArgTrans[\Context']{\typeVec{a_1}{A_1}}.
%                \app{\WTermTrans{\IndD}}{
%                    \IndParamTerm{p_1}{1}}{
%                    \dots}{
%                    \IndParamTerm{p_1}{n}}{
%                    \IndIndexTerm{V_{1,1}}{1}}{
%                    \dots}{
%                    \IndIndexTerm{V_{1,m}}{m}}{
%                    (\app{\ConstructorTrans{c_1}}{\vv{p}}{\vv{a_1}})
%                    }$
%  \newline
%  \dots
%  \newline
%  $\WTermTrans[k]{c}:\Pi\WIndArgTrans[\Context']{\typeVec{a_k}{A_k}}.
%                \app{\WTermTrans{\IndD}}{
%                    \IndParamTerm{p_1}{1}}{
%                    \dots}{
%                    \IndParamTerm{p_n}{n}}{
%                    \IndIndexTerm{\WParamTrans{V_{k,1}}}{1}}{
%                    \dots}{
%                    \IndIndexTerm{\WParamTrans{V_{k,m}}}{m}}{
%                    (\app{\ConstructorTrans{c_k}}{\vv{p}}{\vv{a_k}})
%                    }$
%\end{itemize}
%where
%
%\begin{table}[h!]
%\centering
%\begin{tabular}{r@{\hspace{0.5em}}c@{\hspace{0.5em}}l}
%$\ParamIndTrans{\cdot}{\IndD}$ & $\termDef$ & $\cdot$ \\
%$\ParamIndTrans{\typing{a}{A},L}{\IndD}$ & $\termDef$ &
%$
%\left \{ \begin{matrix*}[l]
%a:\TypeTrans{A},\ParamIndTrans{L}{\IndD}, & {\normalfont \propFun[\IndD] \wedge \neg\ \propFun[A]}\\
%a:\TypeTrans{A},\WTermTrans{a}:\app{\WParamTypeTrans{A}}{a}, \ParamIndTrans{L}{\IndD} & \otherwise
%\end{matrix*}\right .$
%\vspace{1em}\\
%\hline\\
%$\IndParamTerm*{p}{j}$ & $\termDef$ & 
%$
%\left \{ \begin{matrix*}[l]
%a & {\normalfont \typing{a}{A}\wedge\propFun[\IndD] \wedge \neg\ \propFun[A]}\\
%\app{a}{\WTermTrans{a}} & \otherwise
%\end{matrix*} \right .$
%\end{tabular}
%\end{table}

\noindent The exceptional translation is extended with the following rules:
\begin{center}
$\Trans{\IndD}\termDef\ConstructorTrans{\IndD}$ \hspace{1em}
$\Trans{c_{1}}\termDef\ConstructorTrans{c_{1}}$ \hspace{1em}
$\dots$ \hspace{1em}
$\Trans{c_{k}}\termDef\ConstructorTrans{c_{k}}$
\end{center}
\end{Definition}

Now we are able to define how the weakly parametric translation works on inductives.

\begin{Definition}
\label{def:WInductiveDef}
Let be $\IndIntro[k][\varContext_\IndD][\varContext_C]$ a valid inductive block
definition, as given in \defRef{def:InductiveDef}. 
We define the exceptional translation 
$\IndIntro[k'][\varContext_{\ConstructorTrans{\IndD}}][\varContext_{\ConstructorTrans{C}}]$
of the previous block where $k'$ is number of new arguments and:
%\newcommand{\WIndArgTrans}[2][\Context{}]{\ensuremath{[\![#2]\!]^{#1}_{\WTransSymbol\mathtexttt{Ind}}}}
\begin{center}
$\varContext_{\WTermTrans{\IndD}} 
=
\{\WTermTrans[i]{\IndD}: \WIndArgTrans[\cdot]{T} \ |\ \typing{\IndD_i}{T} \in \varContext_\IndD\}
$ \\
$\varContext_{\WTermTrans{C}} 
=
\{\WTermTrans[i]{c}: \WIndArgTrans[\cdot]{T} \ |\ \typing{c_i}{T} \in \varContext_C\}
$
\end{center}
with $\WIndArgTrans{\cdot}$ defined as below.

{\renewcommand{\arraystretch}{1.15}
\begin{tabular}{l@{$\hspace{0.2em}\termDef$\ \ }l}
$\WIndArgTrans{\depArrow{x}{A}{B}}$ & 
        $\left \{\begin{matrix*}[l]
        \depArrow{x}{\TypeTrans{A}}{\WIndArgTrans[\Context,\typing{x}{A}]{B}} & \weaklypropFun[A,B]\\
        \depArrow{x}{\TypeTrans{A}}{
                  \WTermTrans{x}}{\app{\WParamTypeTrans{A}}{\Trans{x}}}{
                  \WIndArgTrans[\Context,\typing{x}{A}]{B}} & \otherwise
        \end{matrix*}\right .$ \\
$\WIndArgTrans{A}$ & $\WParamTypeTrans{A}$
\end{tabular}
}

\noindent The weakly parametric exceptional translation is extended with the following rules:
\begin{center}
$\Trans{\IndD}\termDef\ConstructorTrans{\IndD}$ \hspace{1em}
$\Trans{c_{1}}\termDef\ConstructorTrans{c_{1}}$ \hspace{1em}
$\dots$ \hspace{1em}
$\Trans{c_{k}}\termDef\ConstructorTrans{c_{k}}$
\end{center}
\end{Definition}

\begin{figure}
\begin{CoqDef}%{\par\bfseries\ttfamily\noindent%
\\$\IndIntro[2][\typing{\Indeqw}{\depArrow{A}{\type[]}{x}{\TypeTrans{A}}{y}{\TypeTrans{A}}{
                            \arrow{\app{\Indeqe}{A}{x}{y}}{\Prop}}}][
              \\\hspace*{1em}
              \typing{\Consreflw}{\depArrow{A}{\type[]}{x}{\TypeTrans{A}}{y}{\TypeTrans{A}}{
                              \app{\Indeqw}{A}{x}{x}{(\app{\Consrefle}{A}{x})}}}
              ]{} $
\\$\IndIntro[3][\typing{\Indexw}{\multDepArrow{{A:\type[]},{P:\arrow{\TypeTrans{A}}{\prop}},
                                           {\WTermTrans{P}:\depArrow{H}{\TypeTrans{A}}{
                                            \arrow{\TypeTrans{\app{P}{H}}}{\Prop}}}}{
                              \arrow{\app{\Indexe}{A}{P}}{\Prop}}}][
                \\\hspace*{1em}
                \typing{\Consintrow}{
                \multDepArrow{{A:\type[]},{P:\arrow{\TypeTrans{A}}{\prop}},
                              {\WTermTrans{P}:\depArrow{H}{\TypeTrans{A}}{
                                               \arrow{\TypeTrans{\app{P}{H}}}{\Prop}}},
                              {\ensuremath{x}:\ensuremath{A}},
                              {\ensuremath{H}:\TypeTrans{\app{P}{x}}}}{
                 \\\hspace*{1.8cm} \arrow{\app{\WTermTrans{P}}{x}{H}}{
                        \app{\Indexw}{A}{P}{P}{(\app{\Consintroe}{A}{P}{x}{H})}}}
                 }]{}$
\\$\IndIntro[0][\typing{\Indnatw}{\arrow{\app{\Indnate}}{\Type}}][
        \typing{\ConsOw}{\app{\Indnatw}{\ConsOe}}, 
        \typing{\ConsSw}{\depArrow{n}{\Indnate}{\arrow{\app{\Indnatw}{n}}{\app{\Indnatw}{(\app{\ConsSe}{n})}}}}]$
%Inductive eq$_\WTransSymbol\ (A:\type[]{})\ (x:\TypeTrans{A}): 
%\enDepArrow{y}{\TypeTrans{A}}{\arrow{\app{\ConstructorTrans{\text{eq}}}{A}{x}{y}}{\Prop}}$:=\\
%\hspace*{1mm}|$\WTermTrans{\text{refl}}: \app{\WTermTrans{\text{eq}}}{A}{x}{x}{
%                                (\app{\ConstructorTrans{\text{refl}}}{A}{x})}$

%\vspace{1ex}
%Inductive ex$_\WTransSymbol\ (A:\type[]{})\ 
%(P:\arrow{\TypeTrans{A}}{\prop})\ 
%(\WTermTrans{P}:
%      \enDepArrow{H}{\TypeTrans{A}}{\arrow{\TypeTrans{\app{P}{H}}}{\Prop}}):\\
%\hspace*{2em}\arrow{\app{\ConstructorTrans{\text{ex}}}{A}{P}}{\Prop}$ := \\
%\hspace*{1mm}|$\WTermTrans{\text{intro}}:\depArrow{x}{A}{
%                                     H}{\TypeTrans{\app{P}{x}}}{
%                                     \arrow{\app{\WTermTrans{P}}{x}{H}}{
%                                            \app{\text{\WTermTrans{ex}}}{
%                                             A}{
%                                             P}{
%                                             \WTermTrans{P}}{
%                                             (\app{\ConstructorTrans{\text{intro}}}{A}{P}{x}{H})}}}$
                                             
%\vspace{1ex}
%Ind $\WTermTrans{\mathtexttt{nat}}:\arrow{\ConstructorTrans{\mathtexttt{nat}}}{\Type[]{}} := \\ 
%\hspace*{1mm}|\WTermTrans{\text{O}}: \app{\WTermTrans{\text{nat}}}{\ConstructorTrans{\text{O}}}\\
%\hspace*{1mm}|\WTermTrans{\text{S}}: \depArrow{n}{
%                                              \ConstructorTrans{\text{nat}}}{
%                                              \arrow{\app{\WTermTrans{\text{nat}}}{n}}{
%                                                     \app{\WTermTrans{\text{nat}}}{
%                                                          (\app{\ConstructorTrans{\text{S}}}{   
%                                                               n}})}}$
\end{CoqDef}%}%
    \caption{Examples of Weakly Parametric Translation of Inductive Types}
    \label{fig:IndWTrans}
\end{figure}

This translation cannot be applied over all inductive block definitions. This is because the number of parameters 
of the translation may vary if, for example, when defining two mutual inductive types, one is a proposition 
and the other is a type. Depending on the type of the arguments it is possible that the proposition type receive
less arguments, so the calculus rejects the definition because the parameters must be the same for 
all inductives.
In \figRef{fig:IndWTrans}, we see how the weakly translation works over inductive types.

\begin{Theorem}
For any inductive \IndD{} not using nested inductive types, the \defRef{def:WInductiveDef} is well-typed 
and preserves the positivity criterion.
\end{Theorem}

\begin{Proof}
The well-typed property is a consequence of \theoRef{theorem:typingSoundness}. The conservation of the
positivity criterion is due to the fact that the translation preserves the structure of the constructor.
\end{Proof}

\begin{Corollary}
The weakly parametric translation remains sound when adding the inductives and their constructors.
\end{Corollary}

\newcommand{\MatchTerm}{\ensuremath{Q}}
\begin{Definition}
\label{def:PM}
Assume an inductive \IndD{} as given in \defRef{def:InductiveDef}. Let \MatchTerm{} be the well-typed 
pattern-matching term defined as
\begin{adjustwidth}{1.5em}{0cm} \normalfont \sffamily
match \M{} return $\lambda \app{(\typing{i_1}{I_1})}{\dots}{(\typing{i_m}{I_m})}{
                    (x:\app{\IndD}{X_1}{\dots}{X_n}{i_1}{\dots}{i_m})}.R$ with\\
$|\ \app{c_1}{a_{1,1}}{\dots}{a_{1,l_1}} \Rightarrow N_1$\\
$\dots$\\
$|\ \app{c_k}{a_{k,1}}{\dots}{a_{k,l_k}} \Rightarrow N_k$\\
end
\end{adjustwidth}
where
\begin{center}
{
\def\intercolumn{2.5em}
\def\interline{5pt}
$
\ctyping{\vv{X}}{\vv{P}}
\hspace{\intercolumn}
\ctyping{\vv{Y}}{\subst{\vv{I}}{\vv{p}}{\vv{X}}}
\hspace{\intercolumn}
\ctyping{M}{\app{\IndD}{X_1}{\dots}{X_n}{Y_1}{\dots}{Y_m}}
$
\vspace{\interline{}}\\
$
\ctyping[\vv{i}:\subst{\vv{I}}{\vv{p}}{\vv{X}},x:\app{\IndD}{\vv{X}}{\vv{i}}]{R}{\Type[]{}}
\hspace{\intercolumn}
\ctyping{Q}{\subst{R}{\vv{i}}{\vv{Y}}{x}{M}}
$
\vspace{\interline{}}\\
$
\ctyping[\vv{a_1}:\vv{A_1}]{N_1}{\subst{R}{
                                                    \vv{i}}{\subst{\vv{V_1}}{p}{X}}{
                                                    x}{\app{c_1}{\vv{X}}{\vv{a_1}}}}
$\\
$\dots$
\vspace{\interline{}}\\
$
\ctyping[\vv{a_k}:\vv{A_k}]{N_k}{\subst{R}{
                                                    \vv{i}}{\subst{\vv{V_k}}{p}{X}}{
                                                    x}{\app{c_k}{\vv{X}}{\vv{a_k}}}}
$
}
\end{center}
\end{Definition}

The exceptional translation has a major issue. Singleton elimination must not be considered in order to 
translate pattern-matching terms with impredicativity. Applying the exceptional
translation over an inductive proposition with one constructor yields an inductive proposition with two 
constructors (the default error), which breaks singleton elimination. If we do not consider singleton elimination,
We to proceed with the following.
\newcommand{\WIndexIndTrans}[2][\IndD]{\TypeTrans{#2}_{\WTransSymbol\mathtexttt{Index}}^{#1}}
\begin{Definition}
\label{def:WParamTransPM}
Assume \MatchTerm{} is the pattern-matching defined in \defRef{def:PM}. We define \WParamTrans{\MatchTerm{}}
as:
\begin{adjustwidth}{1.5em}{0cm} \normalfont \sffamily
match \WParamTrans{\M{}} return $\lambda\app{(\WIndexIndTrans{\vv{i}:\vv{I}})}{
                    (x: \app{\Trans{\app{\IndD}{X_1}{\dots}{X_n}}}{\vv{i}_{\mathtexttt{non-param}}})}{$\\
\hspace*{4cm}  $(\WTermTrans{x}: \app{\WParamTrans{\app{\IndD}{X_1}{\dots}{X_n}}}{\vv{i}_{\mathtexttt{all}}})}.
                    \app{\WParamTypeTrans[\Context]{R}}{\Trans{\MatchTerm_x}}$ with\\
$|\ \app{\WTermTrans[1]{c}}{\vv{a}} \Rightarrow \WParamTrans[\Context,\Delta_1]{N_1}$\\
$\dots$\\
$|\ \app{\WTermTrans[k]{c}}{\vv{a}} \Rightarrow \WParamTrans[\Context,\Delta_k]{N_k}$\\
end
\end{adjustwidth}
%where $\MatchTerm_x$ is the following pattern-matching:
%\begin{center}
%{
%\begin{adjustwidth}{1.5em}{0cm} \normalfont \sffamily
%match $x$ return $\lambda \app{(i_1:I_1)}{\dots}{(i_m:I_m)}{
%                    (x:\app{\IndD}{X_1}{\dots}{X_n}{i_1}{\dots}{i_m})}.R$ with\\
%$|\ \app{c_1}{a_{1,1}}{\dots}{a_{1,l_1}} \Rightarrow N_1$\\
%$\dots$\\
%$|\ \app{c_k}{a_{k,1}}{\dots}{a_{k,l_k}} \Rightarrow N_k$\\
%end
%\end{adjustwidth}
%}
%\end{center}
\end{Definition}

\begin{Theorem}[Pattern-matching soundness]
Let \MatchTerm{} be the term defined as \defRef{def:PM}. With \defRef{def:WParamTransPM} and 
$b\termEq\propFun[\subst{R}{\vv{i}}{\vv{Y}}{x}{M}]$ we have:
\begin{center}  
    $\ctyping[][\WContextTrans{\Context}]{\WParamTrans{\MatchTerm}}{
                        \app{\WParamTypeTrans{\subst{R}{\vv{i}}{\vv{Y}}{x}{M}}}{\ParamTrans{\MatchTerm}}}$
\end{center}
\end{Theorem}

\begin{Proof}
This result is a consequence of \theoRef{theorem:typingSoundness}. The return type of the match corresponds
weakly parametrically with each constructor of the inductive type, resulting in a straightforward
application of the theorem.
\end{Proof}

As a immediate consequence of the previous theorem we have:

\begin{Theorem}
The translation preserves $\iota$-rules
\end{Theorem}
\begin{Proof}
The structure of the translation is preseved by the translation.
\end{Proof}

\vspace{1em}
\noindent With the previous definitions it is possible to translate equality and existential as shown in
\figRef{fig:IndWTrans}. Now the proposition:
\begin{center}
$\exists (n:\mathtexttt{nat}),\ (\app{\raiseFun}{\ttUnit}) = n$
\end{center}
admits a weakly parametric translation with the following proof:

\noindent
$(\app{\WTermTrans{\mathtexttt{intro}}}{
       (\app{\mathtexttt{TypeVal}}{\Indnate}{\ExceptionTrans{\Indnat}})}{
\\\hspace*{2em} (\abs{n}{\Indnate}{
                    \app{\mathtexttt{TypeVal}}{
\\\hspace*{6em}     (\app{\Indeqe}{(\app{\TypeVal}{\Indnate}{\ExceptionTrans{\Indnat}})}{
                       (\app{\raiseFun}{\ttUnit})}{n})}{
\\\hspace*{6em}     (\app{\ExceptionTrans{\Indeq}}{(\app{\TypeVal}{\Indnate}{\ExceptionTrans{\Indnat}})}{
                       (\app{\raiseFun}{\ttUnit})}{n})}{
                 }}}{
\\\hspace*{2em} (\abs{n}{\Indnate}{
                    \app{\Indeqw}{(\app{\mathtexttt{Typeval}}{\Indnate}{\ExceptionTrans{\Indnat}})}{
                        (\app{\raiseFun}{\ttUnit})}{n})
                    }}{
\\\hspace*{2em} (\app{\raiseFun}{\ttUnit})}{
\\\hspace*{2em} (\app{\Consrefle}{(\app{\mathtexttt{TypeVal}}{\Indnate}{\ExceptionTrans{\Indnat}})}{
                      (\app{\raiseFun}}{\ttUnit}))}{
\\\hspace*{2em} (\app{\Consreflw}{(\app{\mathtexttt{TypeVal}}{\Indnate}{\ExceptionTrans{\Indnat}})}{
                      (\app{\raiseFun}}{\ttUnit}))
                      }$
                      
Now we have to show that $\WParamTrans{(\app{\raiseFun}{\ttUnit})}$ does not exist. This is true because 
is not possible to define a parametric translation of $\raiseFun$.

\section{Parametric Modality}
