\begin{Theorem}
Given \Context, $M$, and $A$ with \wfContext{\Context} and \ctyping{M}{A}, then we have
\ctyping[][\WContextTrans{\Context}]{\WTrans{M}}{\WTypeTrans{A}}
\end{Theorem}

\begin{Proof}
By a mutual induction over the typing derivation.
\begin{ProofCase}{\CicType}
We need to prove 
$\ctyping[][\WContextTrans{\Context}]{
          \app{\TypeVal}{\type}{\TypeErr}}{
          \type[j]}$. 
We now that \TypeVal{} has type 
$\depArrow{A}{\type[i']}{\arrow{\arrow{\ExceptionType}{A}}{\type[i']}}$ and \type{} has type 
$\Type[j]$ with $i<j$. By a double \CicApp{}, we can conclude the result unifying $j$ with $i'$.
\end{ProofCase}

\begin{ProofCase}{\CicProp}
We need to prove 
$\ctyping[][\WContextTrans{\Context}]{
          \app{\TypeVal}{\Prop}{(\depArrow{A}{\Prop}{A})}}{
          \type}$. 
This is similar to the previous case.
\end{ProofCase}

\begin{ProofCase}{\CicWeak}
We need to prove 
$\ctyping[\typing{a}{\WTypeTrans{A}}][\WContextTrans{\Context}]{
          \WTrans{M}}{
          \WTypeTrans{B}}$. 
By the induction hypothesis we have $\ctyping[][\WContextTrans{\Context}]{
          \WTrans{M}}{
          \WTypeTrans{B}}$. It only remains to show
$\ctyping[][\WContextTrans{\Context}]{\WTypeTrans{A}}{s'}$ for some $s'$. We proceed by 
a case analysis over $s$
\begin{SubProofCase}{$s \equiv \Prop{}$}
We have $\ctyping[][\WContextTrans{\Context}]{\WTrans{A}}{\Prop}$ by the induction hypothesis
and also $\WTypeTrans{A}\equiv\WTrans{A}$. We get the result by applying \CicWeak{} with 
$s'\equiv\Prop$.
\end{SubProofCase}

\begin{SubProofCase}{$s \equiv \Type$}
We have $\ctyping[][\WContextTrans{\Context}]{\WTrans{A}}{\type[s']}$ by the induction hypothesis
and also $\WTypeTrans{A}\equiv\app{\El}{\WTrans{A}}$. In order to conclude, we need that 
$\ctyping[][\WContextTrans{\Context}]{\app{\El}{\WTrans{A}}}{s'}$. Recall that
\El{} has type \arrow{\type}{\Type}. By \CicApp{} on \app{\El}{\WTrans{A}}, we derive it 
types to \Type[s']. It is possible to conclude by \CicWeak{} on the previous results unifying
the universe levels.
\end{SubProofCase}
\end{ProofCase}

\begin{ProofCase}{\CicTypeProd}
We need to prove 
\begin{center}
$\ctyping[][\WContextTrans{\Context}]{
    \app{\TypeVal}{
        (\depArrow{x}{\WTypeTrans{A}}{\WTypeTrans[\typing{x}{A}]{B}}) 
        }{
        (\abs{e}{\ExceptionType}{x}{\WTypeTrans{A}}{\app{\WTransErr[\typing{x}{A}]{B}}{e}})
        }
    }{
    \type[max(i,j)]
    }$. 
\end{center}
If the first argument of \TypeVal{} types to $\type[max(i,j)]$ and the second to 
$(\depArrow{x}{\WTypeTrans{A}}{\WTypeTrans[\typing{x}{A}]{B}})$, we can conclude the proof
with a double application of \CicApp{} rule which gives the desired type. We proceed to 
check the first argument.

\noindent By the induction hypothesis we know that:
\begin{itemize}
\item $\ctyping[][\WContextTrans{\Context}]{\WTrans{A}}{\type}$
\item $\ctyping[\typing{x}{\WTypeTrans{A}}][\WContextTrans{\Context}]{
                \WTrans[\typing{x}{A}]{B}}{\type[j]}$
\end{itemize}
We can derive from these hypothesis the followings:
\begin{itemize}
\item $\ctyping[][\WContextTrans{\Context}]{\WTypeTrans{A}}{\Type}$
\item $\ctyping[\typing{x}{\WTypeTrans{A}}][\WContextTrans{\Context}]{
                \WTypeTrans[\typing{x}{A}]{B}}{\Type[j]}$
\end{itemize}
by expanding the definition of \WTypeTrans{\cdot} and applying \CicApp{} rule
over \El{} and \WTrans{\cdot} in both cases. By \CicTypeProd{} rule we get
\begin{center}
$\ctyping[][\WContextTrans{\Context}]{
    \depArrow{x}{\WTypeTrans{A}}{\WTypeTrans[\typing{x}{A}]{B}}}{
    \Type[max(i,j)]}$
\end{center}
It remains to show that 
\begin{center}
$\ctyping[][\WContextTrans{\Context}]{
    (\abs{e}{\ExceptionType}{x}{\WTypeTrans{A}}{\app{\WTransErr[\typing{x}{A}]{B}}{e}})
    }{
    (\arrow{\ExceptionType}{\depArrow{x}{\WTypeTrans{A}}{\WTypeTrans[\typing{x}{A}]{B}}})
}$
\end{center}
which can be reduced to prove the body of the lambda
\begin{center}
$\ctyping[\typing{e}{\ExceptionType},\typing{x}{\WTypeTrans{A}}][\WContextTrans{\Context}]{
    (\app{\WTransErr[\typing{x}{A}]{B}}{e})
    }{
    \WTypeTrans[\typing{x}{A}]{B}
}$
\end{center}
We have $\app{\WTransErr[\typing{x}{A}]{B}}{e} \equiv \app{\Err}{\WTrans[\typing{x}{A}]{B}}{e}$ 
by definition, and $\typing{\Err}{\depArrow{A}{\type}{\arrow{\ExceptionType}{\app{\El}{A}}}}$.
Applying \CicApp{} two times we get 
\begin{center}
$\ctyping[\typing{e}{\ExceptionType},\typing{x}{\WTypeTrans{A}}][\WContextTrans{\Context}]{
    (\app{\Err}{\WTrans[\typing{x}{A}]{B}}{e})
    }{
    (\app{\El}{\WTrans[\typing{x}{A}]{B}})
}$
\end{center}
and noticing that $\app{\El}{\WTrans[\typing{x}{A}]{B}} \equiv \WTypeTrans[\typing{x}{A}]{B}$
we are able to conclude.

The first and second argument of \TypeVal{} type to expected type which allow us to conclude.
\end{ProofCase}

\begin{ProofCase}{\CicPropProd}
This case is similar to the previous one with the difference that now we have
$\ctyping[][\WContextTrans{\Context}]{\WTypeTrans{A}}{\Prop}$ instead of
$\ctyping[][\WContextTrans{\Context}]{\WTypeTrans{A}}{\Type}$. This holds
because $\WTypeTrans{A} \equiv \WTrans{A}$ 
and $\ctyping[][\WContextTrans{\Context}]{\WTrans{A}}{\Prop}$ holds by the induction
hypothesis. \\
The rest of the proof remains almost unchanged.
\end{ProofCase}

\begin{ProofCase}{\CicImpred}
In this case we need to prove
\begin{center}
$\ctyping[][\WContextTrans{\Context}]{
    (\depArrow{x}{\WTypeTrans{A}}{\WTypeTrans[\typing{x}{A}]{B}})
    }{
    \Prop
    }$. 
\end{center}
because $\ctyping{(\depArrow{x}{A}{B})}{\Prop}$. By the induction hypothesis we have
\begin{itemize}
\item $\ctyping[][\WContextTrans{\Context}]{\WTrans{A}}{\WTypeTrans{s}}$
\item $\ctyping[\typing{x}{\WTypeTrans{A}}][\WContextTrans{\Context}]{
                \WTrans[\typing{x}{A}]{B}}{\Prop}$
\end{itemize}
It is possible to derive $\ctyping[][\WContextTrans{\Context}]{\WTypeTrans{A}}{s}$ by 
\lemmaRef{lemma:wtrans_sort_involutive}.
With the previous hypothesis, the desired result is obtained by a direct application of 
\CicImpred{} rule.
\end{ProofCase}

\begin{ProofCase}{\CicAbs}
We need to prove
\begin{center}
$\ctyping[][\WContextTrans{\Context}]{
    (\abs{x}{\WTypeTrans{A}}{\WTrans[\typing{x}{A}]{M}})
    }{
    (\depArrow{x}{\WTypeTrans{A}}{\WTypeTrans[\typing{x}{A}]{B}})
    }$. 
\end{center}
The induction hypothesis gives us
\begin{itemize}
\item $\ctyping[\typing{x}{\WTypeTrans{A}}][\WContextTrans{\Context}]{
        \WTrans{M}}{\WTypeTrans{B}}$
\item $\ctyping[][\WContextTrans{\Context}]{
        \WTrans{\depArrow{x}{A}{B}}}{\WTypeTrans{s}}$
\end{itemize}
We can derive 
$\ctyping[][\WContextTrans{\Context}]{\WTypeTrans{\depArrow{x}{A}{B}}}{s}$ by  
\lemmaRef{lemma:wtrans_sort_involutive}. Noticing that
\begin{center}
$
\ctyping[][\WContextTrans{\Context}]{\WTypeTrans{\depArrow{x}{A}{B}}}{s}
\equiv
\ctyping[][\WContextTrans{\Context}]{\depArrow{x}{\WTypeTrans{A}}{\WTypeTrans{B}}}{s}
$
\end{center}
the previous hypotheses allow us to obtain the result by an application of \CicAbs{} rule.
\end{ProofCase}

\begin{ProofCase}{\CicApp}
Here we need to prove
\begin{center}
$\ctyping[][\WContextTrans{\Context}]{\WTrans{\app{M}{N}}}{\WTypeTrans{\subst{M}{x}{N}}}$
\end{center}
By an unfolding the translation and \lemmaRef{lemma:wtrans_substitution}
\begin{center}
$\ctyping[][\WContextTrans{\Context}]{
    \app{\WTrans{M}}{\WTrans{N}}}{\subst{\WTypeTrans{B}}{x}{\WTrans{N}}}$
\end{center}
Also, we have as induction hypothesis
\begin{itemize}
\item \ctyping[][\WContextTrans{\Context}]{\WTrans{M}}{\depArrow{x}{\WTypeTrans{A}}{\WTypeTrans{B}}}
\item \ctyping[][\WContextTrans{\Context}]{\WTrans{N}}{\WTypeTrans{A}}
\end{itemize}
Which allows to conclude by the \CicApp{} rule.
\end{ProofCase}

\begin{ProofCase}{\CicConv}
The induction gives us
\begin{itemize}
\item \ctyping[][\WContextTrans{\Context}]{\WTrans{M}}{\WTypeTrans{B}}
\item \ctyping[][\WContextTrans{\Context}]{\WTrans{A}}{\WTypeTrans{s}}
\item \convertible[\WContextTrans{\Context}]{\WTrans{A}}{\WTrans{B}}
\end{itemize}
We know that \ctyping[][\WContextTrans{\Context}]{\WTypeTrans{A}}{s} by 
\lemmaRef{lemma:wtrans_sort_involutive}.
We need to check if \convertible{\WTypeTrans{A}}{\WTypeTrans{B}}.
There are four cases:
\begin{itemize}
\item \convertible[\WContextTrans{\Context}]{\app{\El}{\WTrans{A}}}{\app{\El}{\WTrans{B}}}.
\item \convertible[\WContextTrans{\Context}]{\WTrans{A}}{\WTrans{B}}.
\item \convertible[\WContextTrans{\Context}]{\WTrans{A}}{\app{\El}{\WTrans{B}}}.
\item \convertible[\WContextTrans{\Context}]{\app{\El}{\WTrans{A}}}{\WTrans{B}}.
\end{itemize}
The first and second cases are allowed because of the hypothesis and conversion. The third and fourth are 
impossibles because if the terms have those forms, it is because one belongs to \Prop{} and the other 
to \Type{}, and those sorts are not convertible in the source and target theories.
Now is possible to apply \CicConv{} which gives the result.
\end{ProofCase}

\begin{ProofCase}{\CicIdem}
We need to prove
\begin{itemize}
\item $\ctyping[\typing{x}{\WTypeTrans{A}}][\wfContext{\Context}]{x}{\WTypeTrans{A}}$
\end{itemize}
given
\begin{itemize}
\item $\ctyping[][\wfContext{\Context}]{\WTrans{A}}{\WTypeTrans{s}}$
\end{itemize}
By \lemmaRef{lemma:wtrans_sort_involutive} we have 
$\ctyping[][\WContextTrans{\Context}]{\WTypeTrans{A}}{s}$ which allows us to conclude
by \CicIdem{} rule
\end{ProofCase}

\end{Proof}